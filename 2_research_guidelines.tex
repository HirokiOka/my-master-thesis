\section{研究指針}

本研究の目的は,エンタテインメントの要素を用いてプログラミング初学者のプログラミングに対する抵抗感を減らし,プログラミングに対する興味を高めるである.


初学者向けプログラミング学習システムの目的は多様であり,プログラミングスキルを向上させることだけでなく,論理的思考力を養うこと,プログラミングに対する興味関心を高めることなど様々である.例えばScratchはプログラマの育成ではなく,自分の考えを表現するための手段・思考としてプログラミングを教えるために設計されており,Viscuitは子供や情報処理の専門家を目指さない大人に対して,プログラミングの楽しさを伝えるために設計されている.

本研究ではビジュアルプログラミング言語(VPL)ではなくテキストプログラミング言語(TPL)によって実践的なプログラミングへ初学者を向かわせるとともに,その楽しさを感じさせることを目指す.

初学者がプログラミング学習で挫折してしまう要因はいくつか考えらえる.プログラミングを初めて学ぶ際,初学者は「写経型学習」を経験する場合が多い.岡本らは,プログラミング学習におけるもっとも重要な目的は,個々の宣言的知識を記憶することではなく,それらを整理,統合して有機的に結び付け,利用可能な知識として習得することであり,写経型学習過程は重要であるとしている一方で,その困難性についても指摘し,写経型学習における初学者のつまずきの事例を認知的負荷理論を用いて類型化している\cite{okamoto}.このつまずきにおいて共通しているのは,初学者は「分からない」ことから学習をやめてしまう,すなわちモチベーションの低下から挫折してしまうということである.本研究ではこのモチベーションを下げる心理的要因について考察し,その中でも大きな要因であると思われる3つに焦点を当てた.


1つは「辛さ」である.プログラミング初学者向け学習教材でも,記述されているプログラムを書き写す写経型学習を行い,サンプルアプリケーションの作成などをする.これは学びのある学習ではあるが,単調な作業になり,初学者にとっては苦痛となる場合がある.


2つ目は「煩わしさ」である.初学者はプログラムを記述するのに不慣れであり,単調な写経型学習は面倒だと感じることがある.また特にコンピュータを日常的に使用する習慣がない場合は,コンピュータを起動し,エディタを立ち上げ,学習サイトや学習書籍を開くというプロセスにも気力がいるため,無意識的にプログラミングから遠ざかりがちである.


3つ目は「難しさ」である.プログラミング学習と言ってもその内容は多岐に渡り,あるプログラミング言語の文法だけでなくオブジェクト指向等の概念やライブラリ,エディタ,IDE,VCS等ツール群の用法など多くを学ばなければならない.初学者にとって難解な概念も多く,これらを並行して学ぶことはモチベーションを下げかねない.

上記のモチベーションを下げる要因を軽減するため,以下の指針を設けた.開発するシステムにはこれに基づいた設計指針を設ける.

\begin{enumerate}
  \item {\bf 遊び(エンタテインメント)の要素を取り入れる}

  コンピュータを日頃使わない者にとって,プログラミングに触れる機会は少ないため,初学者はプログラミングをすることあるいはプログラムを読むことに慣れておらず,少なからず抵抗感があると考えらえる.本研究ではエンタテインメントの要素を取り入れることで,プログラムに触れることへの抵抗感を減らすとともに能動的な利用を促し,プログラムに触れることを楽しんでもらうことを目指す.

  \item {\bf 利用のハードルを下げる}

  初学者にとって環境構築や,普段使用しないソフトウェアの導入などの手間は,プログラミングに対するモチベーションを下げる要因となる.本研究ではシステムを利用する際の手間を極力小さくすることで,日常的な利用を促進し,初学者がプログラミングと接する機会を増やすことを目指す.

  \item {\bf 不要な負荷をかけない}

  複雑な概念や理解が何回な要素は初学者の挫折の要因である.本研究では極力前提知識などを少なくし,初学者がプログラムあるいはプログラミングのみに集中できるように設計する.
\end{enumerate}

研究指針と設計した各システムでの設計指針の関係を表\ref{research_guidelines}に示す.

\begin{table}[ht]
  \centering
  \caption{各システムでの設計指針}
  \label{research_guidelines}
    \begin{tabular}{|c|c|c|} \hline
      研究指針 & コードリーディング支援システム & プログラミングゲーム  \\ \hline\hline
      遊びの要素を取り入れる & クイズ・占いを用いる & 
      \begin{tabular}{c}
        駆け引き・アドリブを\\取り入れる
      \end{tabular}\\ \hline
      利用のハードルを下げる & 日常的に使用するツールに組み込む & 見て楽しめる\\ \hline
      不要な負荷を書けない & プログラムを読むことに専念させる & 専門知識を必要としない \\ \hline
      
    \end{tabular}
\end{table}