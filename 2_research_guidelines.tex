\section{デザイン指針}

% 本研究の目的は,エンタテインメントの要素を用いてプログラミングを日常に融け込ませることで,プログラミング初学者のプログラミングに対する抵抗感を減らし,プログラミングに対する興味を高めることである.

% 初学者向けプログラミング学習システムの目的は多様であり,プログラミングスキルを向上させることだけでなく,論理的思考力を養うこと,プログラミングに対する興味関心を高めることなど様々である.例えばScratchはプログラマの育成ではなく,自分の考えを表現するための手段・思考としてプログラミングを教えるために設計されており,Viscuitは子供や情報処理の専門家を目指さない大人に対して,プログラミングの楽しさを伝えるために設計されている.

% 本研究ではビジュアルプログラミング言語(VPL)ではなくテキストプログラミング言語(TPL)によって実践的なプログラミングへの興味喚起を行うとともに,その魅力・楽しさを感じさせることを目指す.

% 初学者がプログラミング学習で挫折してしまう要因はいくつか考えらえる.プログラミングを初めて学ぶ際,初学者は「写経型学習」を経験する場合が多い.岡本らは,プログラミング学習におけるもっとも重要な目的は,個々の宣言的知識を記憶することではなく,それらを整理,統合して有機的に結び付け,利用可能な知識として習得することであり,写経型学習過程は重要であるとしている一方で,その困難性についても指摘し,写経型学習における初学者のつまずきの事例を認知的負荷理論を用いて類型化している\cite{okamoto}.このつまずきにおいて共通しているのは,初学者は「分からない」ことから学習をやめてしまう,すなわちモチベーションの低下から挫折してしまうということである.本研究ではこのモチベーションを下げる心理的要因について考察し,その中でも大きな要因であると思われる3つに焦点を当てた.


% 1つは「辛さ」である.プログラミング初学者向け学習教材でも,記述されているプログラムを書き写す写経型学習を行い,サンプルアプリケーションの作成などをする.これは学びのある学習ではあるが,単調な作業になり,初学者にとっては苦痛となる場合がある.

% 2つ目は「煩わしさ」である.初学者はプログラムを記述するのに不慣れであり,単調な写経型学習は面倒だと感じることがある.また特にコンピュータを日常的に使用する習慣がない場合は,コンピュータを起動し,エディタを立ち上げ,学習サイトや学習書籍を開くというプロセスにも気力がいるため,無意識的にプログラミングから遠ざかりがちである.

% 3つ目は「難しさ」である.プログラミング学習と言ってもその内容は多岐に渡り,あるプログラミング言語の文法だけでなくオブジェクト指向等の概念やライブラリ,エディタ,IDE,VCS等ツール群の用法など多くを学ばなければならない.初学者にとって難解な概念も多く,これらを並行して学ぶことはモチベーションを下げかねない.

% 上記のモチベーションを下げる要因を軽減し,初学者がプログラミング学習に挫折してしまうことを防ぐため,以下の指針を設けた.開発するシステムにはこれに基づいた設計指針を設ける.


%別の案
本研究の目的は,エンタテインメントの要素を用いてプログラミングを日常に融け込ませることで,プログラミング初学者のプログラミングに対する抵抗感を減らし,プログラミングに対する興味を高めることである.

初学者向けプログラミング学習システムの目的は多様であり,プログラミングスキルを向上させることだけでなく,論理的思考力を養うこと,プログラミングに対する興味関心を高めることなど様々である.その中で多く用いられるのはVPLであり,例えばScratchはプログラマの育成ではなく,自分の考えを表現するための手段・思考としてプログラミングを教えるために設計されており,Viscuitは子供や情報処理の専門家を目指さない大人に対して,プログラミングの楽しさを伝えるために設計されている.これらは小・中学校などで実際に教育に導入され,プログラミング教育に貢献している.

しかしながらこれらのVPLによる興味喚起は網羅的とは言い難い.特に,より上の年齢層でプログラミング学習を始めた者などは,実践的な開発で用いられるTPLへのイメージとVPLの乖離から,興味を持てない場合がある.

またTPLによるプログラミングを初めて学ぶ際は,初学者は「写経型学習」を経験する場合が多い.岡本らは,プログラミング学習におけるもっとも重要な目的は,個々の宣言的知識を記憶することではなく,それらを整理,統合して有機的に結び付け,利用可能な知識として習得することであり,写経型学習過程は重要であるとしている一方で,その困難性についても指摘し,写経型学習における初学者のつまずきの事例を認知的負荷理論を用いて類型化している\cite{okamoto}.このつまずきにおいて共通しているのは,初学者は「分からない」ことから学習をやめてしまう,すなわちモチベーションの低下から挫折してしまうということである.

本研究ではTPLを用いつつも,より利用ハードルの低いシステムにより,初学者のプログラミングに対するモチベーションを向上させることを目指した.また初学者がプログラミング学習に挫折してしまうことを防ぐため,以下の指針を設けた.開発するシステムにはこれに基づいた設計指針を設ける.


\begin{enumerate}
  \item {\bf 遊び(エンタテインメント)の要素を取り入れる}
  
  従来の学習システムでは,プログラミングそのものの楽しさを伝えることを重視していたが,本研究ではそれ以前にゲーミフィケーションなどのエンタテインメントの要素を取り入れることで,プログラミング以外の遊びによる楽しさを取り入れる.これにより,プログラムに触れることへの抵抗感を減らすとともに能動的な利用を促し,プログラムに触れることを楽しんでもらうことを目指す.


  \item {\bf プログラムを書かずとも楽しめる}

  % 初学者にとって環境構築や,普段使用しないソフトウェアの導入などの手間は,プログラミングに対するモチベーションを下げる要因となる.本研究ではシステムをプログラミングを行わずとも楽しむことができ,日常的な利用を促進し,初学者がプログラミングと接する機会を増やすことを目指す.

  コンピュータを日頃使わない者にとって,プログラミングに触れる機会は少ないため,初学者はプログラミングに慣れておらず,少なからず抵抗感があると考えらえる.従来のプログラミングすることが前提となっているシステムは,初学者にとって利用の心理的ハードルが高いため習慣的に取り組めない場合がある.本研究ではシステムをプログラミングを行わずとも楽しむことができ,日常的な利用を促進し,初学者がプログラミングと接する機会を増やすことを目指す.

  \item {\bf 初学者と上級者をつなげる}

  % プログラミング学習の促進・興味喚起をするためには,システムを利用してもらうためのモチベーションを持ってもらうことが必要不可欠である.初学者がシステムを利用したくなるきっかけを創出し,継続的な利用を促す.

  プログラミングを学習する際に身近にプログラマがいない場合,学習方法が分からなかったり,分からない部分を質問したりすることができず,挫折してしまう場合がある.また初学者が手本となるプログラマを見つけることは,プログラミングの手法を参考にしたり,学習方法を真似るなどメリットが大きい.本研究では,プログラミング初学者とプログラマが関わりを持つための工夫を施すことにより,初学者のプログラミング学習を促進することを目指す.

\end{enumerate}

デザイン指針と設計した各システムでの設計指針の関係を表\ref{research_guidelines}に示す.

\begin{table}[ht]
  \centering
  \caption{各システムでの設計指針}
  \label{research_guidelines}
    \begin{tabular}{|c|c|c|} \hline
      デザイン指針 & 
      \begin{tabular}{c}
        コードリーディング\\支援システム 
      \end{tabular}
      & プログラミングゲーム  \\ \hline\hline
      遊びの要素を取り入れる & クイズ・占いを用いる & 
      \begin{tabular}{c}
        駆け引き・アドリブを\\取り入れる
      \end{tabular}\\ \hline
      \begin{tabular}{c}
        プログラムを書かずとも\\楽しめる
      \end{tabular}
      & プログラムを読んで楽しめる & 観戦して楽しめる\\ \hline
      \begin{tabular}{c}
        初学者と上級者をつなげる
      \end{tabular}
      & コミュニケーションを創出する &
      \begin{tabular}{c}
        プログラマに対する\\憧れを創出する
      \end{tabular}\\ \hline
      
      
    \end{tabular}
\end{table}