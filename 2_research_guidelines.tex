\section{研究指針}

本研究の目的は,エンタテインメントの要素を用いてプログラミング初学者のプログラミングに対する抵抗感を減らし,プログラミングを楽しんでもらうことである.


初学者向けプログラミング学習システムの目的は多様であり,プログラミングスキルを向上させることだけでなく,論理的思考力を養うこと,プログラミングに対する興味関心を高めることなど様々である.例えばScratchはプログラマの育成ではなく,自分の考えを表現するための手段・思考としてプログラミングを教えるために設計されており,Viscuitは子供や情報処理の専門家を目指さない大人に対して,プログラミングの楽しさを伝えるために設計されている.

本研究ではVPLではなくTPLによって実践的なプログラミングへ初学者を向かわせるとともに,その楽しさを感じさせることを目指す.


初学者がプログラミング学習を挫折してしまう要因はいくつか考えられるが,本研究ではそのうち3つの要素に焦点を当てた.

1つは「辛さ」である.プログラミング初学者向け教材では,記述されているプログラムを書き写す,いわゆる「写経」を行い,サンプルアプリケーションの作成などをする.これは学びの多い作業ではあるが,単調な作業になり,初学者にとっては苦痛を伴う場合がある.


2つ目は「煩わしさ」である.初学者が挫折する原因として多いのが環境構築が上手くいかず,学習を諦めてしまうことである.また特にコンピュータを日常的に使用する習慣がない場合はプログラミングから遠ざかりがちである.


3つ目は「難しさ」である.プログラミング学習と言ってもその内容は多岐に渡り,あるプログラミング言語の文法だけでなくオブジェクト指向等の概念やライブラリ,エディタ,IDE,VCS等ツール群の用法など多くを学ばなければならない.初学者にとって難解な概念も多く,これらを並行して学ぶことはモチベーションを下げかねない.

上記の問題を解決するため,以下の指針を設けた.開発するシステムにはこれに基づいた設計指針を設ける.

\begin{enumerate}
  \item {\bf エンタテインメントの要素を取り入れる}

  コンピュータを日頃使わない者にとって,プログラミングに触れる機会は少ないため,初学者はプログラミングをすることあるいはプログラムを読むことに慣れておらず,少なからず抵抗感があると考えらえる.本研究ではエンタテインメントの要素を取り入れることで,プログラムに触れることへの抵抗感を減らすとともに能動的な利用を促し,プログラムに触れることを楽しんでもらうことを目指す.

  \item {\bf 利用のハードルを下げる}

  初学者にとって環境構築や,普段使用しないソフトウェアの導入などの手間は,プログラミングに対するモチベーションを下げる要因となる.本研究ではシステムを利用する際の手間を極力小さくすることで,日常的な利用を促進し,初学者がプログラミングと接する機会を増やすことを目指す.

  \item {\bf 不要な負荷をかけない}

  複雑な概念や理解が何回な要素は初学者の挫折の要因である.本研究では極力前提知識などを少なくし,初学者がプログラムあるいはプログラミングのみに集中できるように設計する.
\end{enumerate}

研究指針と設計した各システムでの設計指針の関係を表\ref{research_guidelines}に示す.

\begin{table}[ht]
  \centering
  \caption{各システムでの設計指針}
  \label{research_guidelines}
    \begin{tabular}{|c|c|c|} \hline
      研究指針 & コードリーディング支援システム & プログラミングゲーム  \\ \hline\hline
      遊びの要素を取り入れる & クイズ・占いを用いる & 
      \begin{tabular}{c}
        駆け引き・アドリブを\\取り入れる
      \end{tabular}\\ \hline
      利用のハードルを下げる & 日常的に使用するツールに組み込む & 見て楽しめる\\ \hline
      不要な負荷を書けない & プログラムを読むことに専念させる & 専門知識を必要としない \\ \hline
      
    \end{tabular}
\end{table}