\section{まとめ}

本論文では,テキストプログラミング言語によるプログラミング初学者の学習促進・興味喚起をするための2つのアプリケーションを提案した.

1つはクイズ,占いといったエンタテインメントを交えてGitHubにあるソースコードを初学者に読解させることでコードリーディングを促進するものであり,筆者の所属する研究室でケーススタディを行い,得られた問題点を改善したものをWebアプリケーションとして実装し直し,UWW2019にて議論を行った.結果,肯定的な反応が得られ,プログラミング言語への理解が深まったという意見が得られたり,プログラミングにまつわるコミュニケーションを促進できている様子が確認されたが,コードリーディングに対する抵抗感が強くなったといったネガティブな意見もあり,表示プログラムの選択アルゴリズムや継続的な利用を促す工夫など,いくつかの改善点が見つかった.

また2つ目のアプリケーションは,ライブコーディングのようなリアルタイム性・アドリブの要素を取り入れた対人形式のプログラミングゲームである.これを習熟したプログラマにプレイさせ,その対戦を初学者に観戦させることで初学者のプログラミングに対する興味関心を高めることを目指した.初学者を対象に評価実験を行った結果,肯定的な評価が多く得られ,初学者のプログラミングに対する興味喚起に成功したがゲームシステム・デザインに改善すべき点が見つかった.

今後はこの2つのシステムを通して得られた結果・意見をもとに,ScratchやViscuitなどを用いてもプログラミングに対する興味を持てない層,特に高校生・大学生のプログラミング初学者を対象にプログラミングに楽しく取り組める学習コンテンツを拡充し,より広い層を対象にシステムの評価を行う.また本研究の結果を元に,より若い世代に対してのプログラミング学習システムの構築も視野に入れている. 