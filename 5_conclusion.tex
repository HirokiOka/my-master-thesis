\section{まとめ}

本論文では,エンタテインメントの要素を用いてプログラミングを日常に融け込ませることで,プログラミング初学者のプログラミングに対する抵抗感を減らし,興味喚起を行うことを目的とし,研究指針を定め,それに基づいてテキストプログラミング言語によるプログラミング初学者の学習促進・興味喚起をするための2つのアプリケーションを提案した.

1つはクイズ,占いといったエンタテインメントを交えてGitHubにあるソースコードを初学者に読解させることでコードリーディングを促進するものであり,筆者の所属する研究室でケーススタディを行い,得られた問題点を改善したものをWebアプリケーションとして実装し直し,UWW2019にて議論を行った.結果,肯定的な反応が得られ,プログラミング言語への理解が深まったという意見が得られたり,プログラミングにまつわるコミュニケーションを促進できている様子が確認されたが,コードリーディングに対する抵抗感が強くなったといったネガティブな意見もあり,表示プログラムの選択アルゴリズムや継続的な利用を促す工夫など,いくつかの改善点が見つかった.

また2つ目のアプリケーションは,ライブコーディングのようなリアルタイム性・アドリブの要素を取り入れた対人形式のプログラミングゲームである.これを習熟したプログラマにプレイさせ,その対戦を初学者に観戦させることで初学者のプログラミングに対する興味関心を高めることを目指した.初学者を対象に評価実験を行った結果,肯定的な評価が多く得られ,初学者のプログラミングに対する興味喚起に成功したがゲームシステム・デザインに改善すべき点が見つかった.


これらの結果から,本研究において定めた研究指針を基にした提案システムは,初学者のプログラミングに対する学習モチベーションを向上させることができていたと言える.しかし,調査対象の規模が小さいことから,どの要素が効果的であったのかや,研究指針の「初学者と上級者をつなげる」という部分が達成できていたかには疑問が残る.今後は見つかった問題点を改善し,より多くのプログラミング初学者を含むプログラマを対象に綿密な調査を行うとともに,どの要素が興味喚起に効果的であったのかや研究指針を基にプログラミング学習コンテンツに必要な要件を明確にする.