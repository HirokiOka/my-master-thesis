\section{関連研究}

\subsection{読書に関する研究}

読書をテーマにした研究は今までに数多く行われてきたが,その中でも紙の書籍における読書と電子書籍における読書を比較する研究は多く行われている.
柴田らの研究では,業務内容などが記載されたマニュアルから答えを探し出す読みを紙の書籍と電子書籍端末において比較している\cite{shibata}.
この研究では,テキストマニュアルから答えを探す読みを行った際に,紙の書籍を用いた方が60.2\%速く答えを得られるという実験結果を得ている.
さらにDelgadoらのメタ分析においては,ディスプレイにおける文章と印刷された文章を比較した結果,デジタルディスプレイの方が文章の理解度が低くなると結論づけられている\cite{pablo}.
また高野らの研究では,電子書籍と紙の書籍で小説を読書した際に,主観的にどちらの媒体が好みであるかを被験者に評価させ,紙の書籍が最も評価が高かったという結果を得ている\cite{takano}.
これらの研究結果から,電子書籍は今後発展し利便性を増す可能性を秘めつつも,紙の書籍における読書には多くのメリットがあり,主観的にも好まれていることから今後も紙の書籍に対する需要は絶えないと予測できる.

また読書をテーマとする研究において,紙の書籍の読書を拡張するという目的のものも行われている.
稲川らは,既存の読書に別の情報を付加し,従来の読書を拡張させる研究を行っている\cite{inagawa1}.この研究では,センサを用いて紙の書籍のページを取得し,ページに応じた音声や映像を流すという手法で読書体験の拡張を図っている.
翌年には,読書中の情報提示の方法に関する評価も行っている\cite{inagawa2}.
この研究では,紙の書籍の読書中に,読んでいるページに関連する画像を読者に表示し,表示する媒体として据置型ディスプレイに表示する場合とHMDに表示する場合を比較しており,画像を表示するタイミングについても,自動的に表示される手法と読者がページをめくる時に表示する半自動的な手法とを比較している.

これらの研究では,視覚あるいは聴覚を用いた情報付与により読書を拡張しているが,読む本のデータとそれに応じた音声,映像データを予め用意しておく必要がある.本研究では,既存のあらゆる書籍に対応して読書を拡張することを目標としているため,これらの研究では本研究の研究目的を実現できない.


\subsection{固定プロジェクタを用いた研究}

近年,プロジェクタを安価に入手できるようになったことや,その急速な発展からプロジェクタを用いた研究は盛んであり,メディアアートやインタラクティブアート,MR,HCIの分野などで主に使用されている.
橋本らの研究では使用者が壁に対してボールを投げる,あるいは投げる動作をとった際に,その動作に応じて壁が崩れるCGを壁に重畳するインタラクティブなプロジェクションマッピングを行っている\cite{hashimoto}.
また磯垣らは1台のプロジェクタと48枚の可動鏡を用いたインタラクティブアート作品Ptolemyを製作している\cite{isogaki}.この作品は,マウスと画像認識等の入力インタフェースにより,可動鏡と投影コンテンツを操作し,1台のプロジェクタによって複数の位置にコンテンツを投影することのできるものである.
土井らはプロジェクションマッピングによる情報提示を用いて,ユーザを支援するデバイスを提案している\cite{doi}.この研究では,箏の弦にプロジェクションマッピングをすることで,その奏法や演奏タイミングの支援を行っている.
また本型の投影面へ情報投影を行っているGuptaらの研究がある\cite{gupta}.この研究では,大型の白紙の本を投影面とし,カメラとプロジェクタを用いたデバイスにより,そのページのめくり具合から,投影するコンテンツを変化させている.

これらの研究から,プロジェクタを用いることで現実の拡張や既存のサービスの利便性を向上させられると考えられる.しかし,大型の固定プロジェクタを用いるシステムは,使用できる状況が限定的であると考えられ,本研究で提案するシステムのような使用場所を選ばないウェアラブルデバイスとしての実現には,取り入れることのできる要素はあるが,そのまま利用するのは難しい.

\subsection{モバイルプロジェクタを用いた研究}

近年のプロジェクタの小型化に伴い,モバイルプロジェクタを用いた研究が多く行われている.
鈴木らは,モバイルプロジェクタとウェブカメラ,PCから成るデバイスを開発している\cite{suzuki}.
この研究では,データベースに格納された既知の紙の文章に対し,プロジェクタから情報を投影することにより,電子文書のようなサービスを実現している.
太田らは,装着型プロジェクタと可動鏡を用いることにより,周辺の状況により投影面を変更することのできる映像投影法を提案している\cite{ota}.
また,プロジェクタをウェアラブルデバイスとして用いた一例として,MistryらのWUWやHarrisonらのOmniTouchがある\cite{pranav,chris}.
これらはカメラとモバイルプロジェクタからなるデバイスであり,周辺の壁や手などに情報投影を行い,カメラによって手のジェスチャを認識することでデバイスへの入力を可能としたウェアラブルデバイスである.
Winklerらは,プロジェクタフォンというプロジェクタとしての機能を実装したスマートフォンを用いている\cite{christian}.この研究で提案されているのは,ショッピングモールでプロジェクタフォンを腰に装着し床にユーザが行きたい店へのナビゲーションを表示することで,屋内での活動を補助するデバイスである.
狩塚らが行った研究では,モバイルプロジェクタを用いて複合現実感を実現するウェアラブルデバイスが提案されている\cite{karitsuka}.いずれのデバイスもモバイルプロジェクタを用いて,従来のサービスを拡張するデバイスであるが,本研究で提案する読書を拡張するというシチュエーションを想定した研究は筆者の知る限りでは存在しない.

