\section{評価実験}

本研究では,プロジェクションマッピングにより読者の必要な情報を紙面に表示することで読書体験を拡張するシステムを提案した.このシステムを使用する際,ユーザが便利だと感じるか,適切に情報付与を行えているかを評価することが必要である.本章では,提案システムを実装したデバイスを用い,被験者に実際に紙の書籍で読書をさせることでその操作性や利便性,文字・画像の表示方法の適切さを評価する実験を行い,その結果を考察する.

\subsection{実験内容}

この実験では,4章で実装したシステムを被験者に装着させ,読書をさせる.読書前にデバイスの操作方法の説明,LeapMotionによる入力ジェスチャの練習を約5分間行う.その後被験者はデバイスの機能を利用しながら各々のペースで紙の書籍の小説を約20ページ読む.その際に,Wikipediaでの語句の意味の検索機能とgoogleでの関連画像検索機能をそれぞれ最低でも1回以上使用させる.そして読書の終了後にデバイスの操作感や情報付与の方法についてアンケートにより調査する.項目は15項目あり,内容は以下のようになっている.読書中の手元の書籍の紙面は,アンケートの結果を考察するため,ビデオカメラによって記録した.

\subsection{実験結果}

実験のアンケート調査で得られた評価を表\ref{survey}に示す.
この結果によると,プロジェクションマッピングしている文字の表示方法に関して,文字の表示位置(質問1)に対しての意見は賛否が分かれる結果となった.文字の大きさ(質問3)については適切であると答える被験者が多かったが,被験者Bと被験者Dは適切でないと回答した.しかし文字の明るさに関しては平均値が4近くであり,概ね適切であると回答する被験者が多かった.
また画像の表示位置(質問5)についても賛否が分かれる結果となった.表示画像の大きさ(質問7)に関しては,全体として平均値が2近くの評価が得られ,あまり適切ではなかったと考えられる.しかし画像の明るさ(質問8)については,文字の明るさと同様概ね適切であるという意見が得られた.
また,文字や画像の表示タイミングについてはあまり良い結果は得られなかった.なお,プロジェクタの照射範囲(質問10)に関しては,平均値が3.8と概ね適切であるという結果が得られた.またデバイスの利便性(質問11)に関しては,3以下の評価が多く,これは質問13のデバイスの操作性に関しての評価が低かったこととも関連があると推察される.


\subsection{考察}

アンケートでの質問の回答についてそれぞれ考察する.
まず質問1については,トラッキングの精度による影響が大きかった.トラッキングが上手くいっている場合では,表示位置が適切であったという意見が得られた反面,トラッキングがずれていた場合では本の文章の文字と投影している文字とが重なり,文字が読みづらかったという意見が得られた.なお文字が重なってしまっているが,投影している文字が明るいため,読みづらさは感じなかったという意見もあった.しかし低い評価の理由には,トラッキングの影響以外にも,プロジェクタの照射範囲が本のページよりも狭く,投影する文字が見切れてしまっているという意見もあった.さらに語句を検索し,名詞を選択する際にデバイスの構造上左側に表示された名詞が選択しづらく,より右側に名詞を表示すべきだという意見も得られた.次に質問3の文字の大きさについては,適切であるという評価が多かった中で被験者Bと被験者Dが適切でないと答えたが.本実験では,投影する文字の大きさを書籍の文章よりも少し大きい程度の大きさで表示したため,投影文字が小さいと感じたのだと考えられる.質問5については,これもトラッキングの精度の影響があると考えらえる.トラッキングが上手くいっている被験者は文章と画像が被らずに見やすかったと回答していた.トラッキングがずれていた被験者に関しては,文字と画像が被っていたり,本の中央に画像が表示され見づらかったと回答していた.なおトラッキングが上手くいっていた場合でも,画像が見切れてしまい見づらかったという意見があった.質問7の回答に関しては,文字と画像が被ることを避けるために画像をやや小さめに表示したことで画像が見づらくなり,評価が低くなったと考えられる.
質問9に関しては,データの処理に時間がかかっているため,名詞を選択してから名詞の意味・関連画像を表示するまでの間にラグがあり,評価が低くなってしまったと考えられる.質問11については全体的に平均またはそれ以下の評価をつける被験者が多かった.その理由として主にあげられたのは,ジェスチャによる操作が難しく,慣れが必要であったため,デバイスの機能を十分に使用できなかったというものであった.デバイス自体が大きく,軽量化すべきだという意見もあった.しかし,その装着感は気になるほどではなかったという意見も得られた.逆に肯定的な意見も多く,分からない単語をすぐに調べられるのは便利であるという意見もあった.質問13に関しては,LeapMotionによる操作性の悪さを理由に低い評価を下している被験者が多かった.多く寄せられた意見として,ジェスチャを認識させるのが難しく,特にswipeジェスチャを上手く行えないというものがあった.LeapMotionが意図せずジェスチャを認識してしまったり,key tapジェスチャとswipeジェスチャを誤認識してしまうという場合もあった.デバイスにLeapMotionが斜めに設置されているため,操作が直感的でないという意見もあった.なおジェスチャをする際は,必然的に片手で本を持たないといけないため,それが負担になったり,読書を妨害していると感じる被験者もいた.
また感想としては,入力操作をスムーズに行うことができトラッキングの精度が上がれば嬉しい,調べる癖がついて良いなどの意見が得られた.一方でジェスチャによる入力は向いていない,片手で本を持つのは疲れるという意見があった.


\subsection{実験のまとめ}

本実験では,提案システムを実装したデバイスを被験者に装着させ紙の小説の読書を行わせた.実験の際には,実装されているWikipediaでの語句検索機能とGoogleでの画像検索機能をそれぞれ1回以上使用させた.その結果,表のような結果が得られ,カメラによるトラッキング精度の低さやLeapMotionによる操作の難しさなど,デバイスに実装している機能の改善すべき点が複数見つかった.しかし,ユーザビリティを向上させられれば,より利便性の高い読書を実現できるという意見を得られたため,トラッキング手法や入力インタフェースの改善を行い,デバイスのユーザビリティを向上させることが必要だと分かった.