\section{研究指針}

本研究の目的は,エンタテインメントの要素を用いて初学者のプログラミングに対する抵抗感を無くし,プログラミングを楽しんでもらうことである.

本研究を遂行するにあたり,以下の指針を設け,開発するシステムにこれに基づいた設計指針を設ける.

\begin{enumerate}
  \item {\bf エンタテインメントの要素を取り入れる}

  コンピュータを日頃使わない者にとって,プログラミングに触れる機会は少ないため,初学者はプログラミングをすることあるいはプログラムを読むことに慣れておらず,少なからず抵抗感があると考えらえる.本研究ではエンタテインメントの要素を取り入れることで,プログラムに触れることへの抵抗感を減らすとともに能動的な利用を促し,プログラムに触れることを楽しんでもらうことを目指す.

  \item {\bf 利用のハードルを下げる}

  初学者にとって環境構築や,普段使用しないソフトウェアの導入などの手間は,プログラミングに対するモチベーションを下げる要因となる.本研究ではシステムを利用する際の手間を極力小さくすることで,日常的な利用を促進し,初学者がプログラミングと接する機会を増やすことを目指す.

  \item {\bf 不要な負荷をかけない}

  複雑な概念や理解が何回な要素は初学者の挫折の要因である.本研究では極力前提知識などを少なくし,初学者がプログラムあるいはプログラミングに集中できるように設計する.
\end{enumerate}