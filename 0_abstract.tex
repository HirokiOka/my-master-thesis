\begin{center}
  {\Large エンタテインメントを用いた\\プログラミング初学者の\\学習意欲促進システムに関する研究}\\
  \vspace{20truept}
    \large{所属専攻・コース:グローバル文化・情報コミュニケーション\\}
    \large{氏名:岡 大貴\\}
    \large{指導教員氏名:西田 健志\\}
\end{center}

\section*{内容梗概}

プログラミングが重要視されている昨今では,プログラミングの授業が小学校に導入されるなど,プログラミングの一般教養化が推し進められている.しかしプログラミングを楽しむためにはある程度の習熟を必要とするため,学習中に挫折してしまうプログラミング初学者も少なくない.プログラミングの楽しさを実感させるための初学者向けプログラミング学習コンテンツも存在するが,その多くは小・中学生を対象として設計されておりそれ以上の年代に対しては効果が薄い場合がある.またその多くがビジュアルプログラミング言語を用いているため,実践的なソフトウェア開発で用いるテキストプログラミング言語とは乖離がある場合もある.なおこれらの学習システムは,学習者が自らプログラミングを行うことで興味喚起を行うため,初学者にとって利用の心理的ハードルが高くなってしまう.テキストプログラミング言語を独習する場合であっても,難解な概念が多く,質問する相手がいないなど問題は多い.

本研究では,これらの問題を解決し,より初学者の負担の小さい学習支援・興味喚起をするため,3つのデザイン指針を設け,それに基づいたシステムを開発した.1つ目は「遊び(エンタテインメント)の要素を取り入れる」であり,これによりプログラミングに習熟していなくとも楽しさを感じさせる.2つ目は「プログラムを書かずとも楽しめる」であり,これによりシステムを利用する心理的ハードルを下げるとともにプログラミングに接することを習慣化する.3つ目に「初学者と上級者をつなげる」ことでコミュニケーションを促し,学習の手本となるプログラマと接する機会を創出する.これらに基づき2つのアプリケーションを開発した.

1つ目はエンタテインメントの要素を取り入れることによりコードリーディングを促進するソースコード閲覧システムである.このシステムではクイズ・占いといったエンタテインメントを交えてGitHubにあるソースコードを表示し,ユーザがそれを読み解くことにより,コードリーディングを楽しみ,プログラムに関するコミュニケーションを促進することを目指した.初めにプロトタイプとしてSlackのbotを開発し,ケーススタディを行った結果,クイズに関しては楽しかったという反応が得られたが,ややプログラミング初学者にとっては難解であるという意見などが得られた.この結果を鑑みて,問題点を改善したシステムをWebアプリケーションとして再実装し,ディスカッションを行った.その結果,コミュニケーションを促進できている様子が観察できたが,表示するソースコードを選択するアルゴリズムの改善や,日常的な使用を促すために更なる工夫が必要だと分かった.

2つ目はリアルタイム性・アドリブを取り入れた対人形式のプログラミングゲームである.このシステムでは習熟度の高いプログラマ2人がプログラミングスキルによって力量を競って勝負することができ,プログラミング初学者でも観戦して楽しむことができる.このゲームの観戦によってプログラミングへの好奇心を高めるとともに,習熟したプログラマへの憧れを創出することで,初学者のプログラミングに対する興味関心を高めることを目指した.評価実験では実際に2人のプログラマの対戦を初学者に観戦させ,プレイヤであるプログラマと観戦した初学者の双方にアンケート調査を行った.実験結果としてゲームのプレイ・観戦は楽しく,プログラミングに対する興味が高まったなどシステムに対する肯定的な意見が多く得られたが,ゲームシステム・デザインの改善点に関するコメントも多く得られた.またプログラマがプレイ時に記述したプログラムを分析した結果,戦略の多様化・より可読性の高いプログラムの表示などいくつかの改善すべき点が見つかった.

これらの結果から,プログラミング初学者に対し,エンタテインメントの要素を用いた学習意欲促進システムを用いることは,プログラミング学習をする上でモチベーションを高めるための有効な方策であることが分かった.今後は提案システムを改善すると共に,より多くのプログラミング初学者を含むプログラマにシステムを使用させ,フィードバックを得ることにより,より良いシステムの構築を目指す.