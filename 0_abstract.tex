\section*{内容梗概}

% 電子書籍リーダの普及により,ディスプレイ上での利便性の高い読書が実現されている.それに対し,紙の書籍での読書は依然として形を変えていないにも関わらず根強い人気があり,この利便性を向上させる機能を実現することができれば,既存の読書体験を拡張することができると考えられる.
% そこで本研究では,紙の書籍を読書中のユーザが必要とする語句の意味やイメージ画像の情報を表示することにより,紙の書籍における読書を拡張するシステムを提案する.現実に情報を付与し拡張する手法には,頭部装着型ディスプレイ(HMD:Head Mounted Display)を用いてディスプレイ上に情報を付与する方法とプロジェクタなどを用いて現実に情報を重畳する手法などが考えられるが,HMDは読書への妨害感,現実への遮蔽感を与える可能性がある.そこで本研究では,カメラ,モバイルプロジェクタ,ハンドジェスチャ認識デバイスからなるペンダント型のウェアラブルデバイスを作成し,読書中の本の名詞を抽出して任意の名詞の意味あるいは関連画像を紙面の余白にプロジェクションマッピングすることにより読書を拡張するシステムを提案する.その際,ハンドジェスチャ認識デバイスを用いた入力インタフェースによりユーザからの入力を可能とした.評価実験では,被験者に提案システムを実装したデバイスを装着して読書させ,デバイスの機能や操作感に関するアンケート調査を行った.実験結果から,デバイスの機能は便利であるという意見も得られたが,それ以上に操作性や情報の表示位置に問題があることが分かった.この結果を受け,よりデバイスの精度,操作性を高めることで,ユーザビリティを向上させることが必要であると分かった.


