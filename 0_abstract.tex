\begin{center}
  {\Large エンタテインメントを用いた\\プログラミング初学者の\\学習意欲促進システムに関する研究}\\
  \vspace{20truept}
    所属専攻・コース:グローバル文化・情報コミュニケーション\\
    氏名:岡 大貴\\
    指導教員氏名:西田 健志\\
\end{center}

\section*{内容梗概}

プログラミングが重要視されている昨今では,プログラミングの授業が小学校に導入されるなど,プログラミングの一般教養化が推し進められている.しかしプログラミングを楽しむためにはある程度の習熟を必要とし,学習中に挫折してしまうプログラミング初学者も少なくない.プログラミングの楽しさを実感させるための初学者向けプログラミング学習コンテンツも存在するが,その多くは小・中学生を対象として設計されておりそれ以上の年代に対しては効果が薄い場合がある.またその多くがビジュアルプログラミング言語を用いているため,実践的なソフトウェア開発で用いるテキストプログラミング言語とは乖離がある場合もある.

そこで本研究ではテキストプログラミング言語を用いつつも,プログラムを書かずとも楽しめる初学者向けの学習・興味喚起を目的とした2つのアプリケーションを開発した1つ目はエンタテインメントの要素を取り入れることによりコードリーディングを促進するソースコード閲覧システムである.このシステムではクイズ,占いといったエンタテインメントを交えてGitHubにあるソースコードを表示し,ユーザが読解それを読み解くことにより,楽しみながらコードリーディング・プログラムに関するコミュニケーションを促進することを目指した.実装したアプリケーションを使用し運用を行い,その後アンケート調査・ディスカッションを行った結果,クイズに関しては楽しかったという回答が得られたが,ややプログラミング初学者にとっては難解であるという意見が得られた.また表示するソースコードを選択するアルゴリズムの改善や,日常的な使用を促すために更なる工夫が必要であり,これらの問題点を今後改善し,より多くの初学者を対象にワークショップ等を行う予定である.

2つ目はリアルタイム性・アドリブを取り入れた対人形式のプログラミングゲームである.このシステムでは習熟度の高いプログラマ2人がプログラミングスキルによって力量を競って勝負することができ,プログラミング初学者でも観戦して楽しむことができる.このゲームの観戦によってプログラミングへの好奇心を高めるとともに,習熟したプログラマへの憧れを創出することで,初学者のプログラミングに対する興味関心を高めることを目指した.評価実験では実際に2人のプログラマの対戦を初学者に観戦させ,プレイヤであるプログラマと観戦した初学者の双方にアンケート調査を行った.実験結果としてゲームのプレイ・観戦は楽しく,プログラミングに対する興味が高まったなどシステムに対する肯定的な意見が多く得られたが,ゲームシステム・デザインの改善点に関するコメントも多く得られた.またプログラマがプレイ時に記述したプログラムを分析した結果,戦略の多様化・より可読性の高いプログラムの表示などいくつかの改善点が見つかった.

この結果を受け,今後は提案システムを改善すると共に,より多くのプログラミング初学者を含むプログラマにシステムを使用させ,より良いシステムの構築を目指す.