\section*{内容梗概}

プログラミングが重要視されている昨今では,プログラミングの授業が小学校に導入されるなど,プログラミングの一般教養化が推し進められている.しかしプログラミングを楽しむためにはある程度の習熟を必要とする.プログラミング初学者でもプログラミングの楽しさを感じられるように学習コンテンツをデザインできれば,初学者の学習モチベーションを損ねることなく,プログラミングをさせることができると考えられる.

そこで本研究では初学者の継続的なプログラミング学習を促すため,2つのアプリケーションを開発した.1つ目はエンタテインメントによりコードリーディングを促進するシステムである.このシステムではクイズ,占いといったエンタテインメントを交えてGitHubにあるソースコードを表示し,ユーザに読解させることにより,楽しみながらコードリーディングを促進することを目指した.実装したアプリケーションを使用し,筆者の所属する研究室のメンバーを対象に運用を行い,その後アンケート調査を行った結果,クイズに関しては楽しかったという回答が得られたが,ややプログラミング初学者にとっては難解であるという意見が得られた.また日常的な使用を促すために更なる工夫が必要であり,これらの問題点を今後改善していく予定である.
2つ目はライブコーディングの要素を取り入れた対人形式のプログラミングゲームである.このシステムでは習熟度の高いプログラマがそのプログラミングスキルによって力量を競うことができ,プログラミング初学者でも観戦して楽しむことができるゲームの作成を通して,初学者のプログラミングに対する興味関心を高めることを目指した.評価実験では実際に2人のプログラマの対戦を初学者に観戦させ,プレイヤであるプログラマと観戦した初学者の双方にアンケート調査を行った.実験結果としてシステムのプレイ・観戦は楽しく,プログラミングに対する興味が高まったなどシステムに対する肯定的な意見が多く得られたが,ゲームシステム・デザインの改善点に関するコメントも多く得られた.この結果を受け,今後は提案システムを改善すると共に,より多くのプログラミング初学者を含むプログラマにシステムを使用させ,より良いシステムの構築を目指す.