\section{はじめに}


近年,Kindleを代表とする電子書籍リーダが普及しつつある.その理由として,紙の書籍のように多くの書籍を所持しなくて済むという点や電子書籍リーダによって実現される便利な機能のおかげで,快適に読書が楽しめるという点がある.電子書籍リーダには指定した語句を辞書検索する機能やメモ機能など読者にとって有益な情報を表示することにより,読書の利便性を高める機能が多数備えられている.一方,紙の書籍による読書は電子書籍リーダにおける読書と比較すると依然としてその形を変えていないにも関わらず,その質感や読了後の達成感を実感できる点などから根強い人気がある.

紙の書籍における読書と電子書籍における読書を比較した研究は多数存在し,例えば高野らの研究では,小説を読む場合に,電子書籍で読む場合よりも紙の書籍で読んだ場合の方が読書の速度が速いという研究結果を得ている\cite{takano}.その他の研究でも紙の優位性を示す結果が多く得られており,紙の書籍の需要は今後も途絶えないと考えられる.しかし紙の書籍では,電子書籍のような語句検索やメモ機能の利用はできない.したがって紙の書籍で読書する際,分からない単語があった場合やメモを書きたい場合には,読書を中断して辞書や電子端末で調べたり,ノートを開いたりする必要があるため,そのたびに読書が阻害される.これらの問題を解決し,紙の書籍による読書を便利にできれば,より快適で拡張された読書が実現されると考える.

そこで本研究では,紙の書籍での読書中に読者の視界に必要な情報を表示することにより,読書自体の形態を変えず,紙の書籍における読書の良さを損ねることなく拡張するシステムを提案する.このシステムを実装することにより,読者は読書中に辞書や電子端末を使用することなく,必要な情報を得られる.読者の視界に情報を表示する方法は大きく分けて2つ考えられ,頭部装着型ディスプレイ(HMD)を用いてディスプレイ上に情報を表示し拡張現実,あるいは複合現実感として実現する方法とプロジェクタなどを使い現実に情報を投影する方法がある.前者では,読書時におけるヘッドギアの装着は読書の妨害感や現実との遮蔽感を生む可能性がある.よって本研究では後者のプロジェクタを用いて現実に情報を投影する方法を採用する.提案システムでは,ウェブカメラとプロジェクタ,PC,ハンドジェスチャ認識デバイスから成るデバイスを使用し,紙の書籍の余白にプロジェクションマッピングによって情報を表示し,読書を拡張する.近年のプロジェクタの発展と低価格化により,小型のモバイルプロジェクタが普及してきているため,本研究ではこのモバイルプロジェクタを用いることで,デバイスを首からかけるペンダント型のウェアラブルデバイスとした.首かけ式にすることで,場所を選ばずにデバイスを使用できる.このデバイスにより,紙の書籍における読書作業を妨げることなく,読者の必要な情報を提供し,既存の読書を拡張できるのではないかと考える.提案システムの実装にあたり,デバイスの操作感や情報提示の方法などが適切かを,デバイスを用いた読書を被験者に行わせることにより評価した.


本論文では以降,2章で関連研究について述べ,3章で提案システムの構成を説明する.4章で提案システムの実装について述べ,5章でその評価実験を行う.最後に6章で本論文をまとめる.

