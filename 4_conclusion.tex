\section{まとめ}

本論文では,プログラミング初学者のプログラミング学習を促進するための2つのアプリケーションを提案した.

1つはクイズ,占いといったエンタテインメントを交えてGitHubにあるソースコードを読解することでコードリーディングを促進するものであり,筆者の所属する研究室でケーススタディを行い,得られた問題点を改善したものをWebアプリケーションとして実装し直し,UWW2019にて議論を行った.結果肯定的な反応が得られ,プログラミング言語への理解が深まったという意見が得られたり,プログラミングにまつわるコミュニケーションを促進できている様子が確認されたが,コードリーディングに対する抵抗感が強くなったという意見もあり,表示プログラムの選択アルゴリズムや継続的な利用を促す工夫など,改善点がいくつか見られた.

2つ目はライブコーディングの要素を取り入れた対人形式のプログラミングゲームを初学者に観戦させることで初学者のプログラミングに対する興味関心を高めることを目指したWebアプリケーションであり,初学者を対象に評価実験を行った結果,肯定的な評価が得られたがゲームシステム・デザインに改善すべき点が見つかった.

今後はこの2つのシステムを通して得られた結果・意見をもとに,ScratchやViscuitなどを用いてもプログラミングに対する興味を持てない層,特に高校生・大学生のプログラミング初学者を対象にプログラミングに楽しく取り組める学習コンテンツを作成し,システムの評価を行う.さらにアンケート調査に止まらない興味推定手法を検討し,延いてはプログラミング学習コンテンツ作成・プログラミングに関する講義のデザインにおけるガイドラインの作成を目指す.