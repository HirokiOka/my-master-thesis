\section{まとめ}

% 本論文では,プロジェクションマッピングによる情報提示を行うウェアラブルデバイスにより,紙の書籍による読書体験を拡張するシステムを提案した.提案システムはウェブカメラから取得した紙面の画像を取得し,光学文字認識により紙面の文章データを得る.そして取得した文章を形態素解析により形態素に分け,名詞のみを抽出し,紙面に表示する.ユーザが紙面に表示された名詞を選択すると,その名詞の意味または関連画像が検索され,紙面上に表示される.このシステムを用いて読書することにより,読者は分からない名詞が出てきた際にも,読書を中断して辞書や電子書籍で調べることなく調べ物が可能で,より拡張された読書を体験できると考えられる.
% 提案システムが,実際に読書の利便性を高められているか,実装した機能について調査した.その結果,デバイスの機能に関しては便利だという意見が得られたが,文字・画像の表示位置,トラッキングの精度と入力インタフェースに改善すべき点が見つかった.
% 今後の課題として,トラッキング精度を改善し,より適切な位置へ情報を投影することが必要である.入力インタフェースとして用いたLeapMotionが使いづらいという意見が多く寄せられたため,読書行為を阻害しないインタフェースの検討が必要である.なおデバイスの情報処理に時間がかかり,文字・画像の投影タイミングが遅いという結果も得られたため,プログラム内容の簡略化やより高速な言語を使用してのシステム開発が必要だと考えられる.
