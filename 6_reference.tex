\begin{thebibliography}{99}
	
	%thesis
	%コードリーディング関連
	\bibitem{nagano}永野真知, 早瀬康裕, 駒水孝裕, 北川博之: GitHubとStackOverflowにおけるユーザ行動の統一的な分析, 情報処理学会第79回全国大会, pp. 363–364 (Mar. 2017).
	\bibitem{shibatou}柴藤大介, 有薗拓也, 宮崎章太, 矢谷浩司: CodeGlass:GitHubのプルリクエストを活用したコード断片のインタラクティブな調査支援システム, 情報処理学会インタラクション, vol.2019, pp.159–16 (Mar. 2019).
	\bibitem{ichinose}一ノ瀬智浩, 畑秀明, 松本健一: ソースコード上の技術的負債除去を活性化させるゲーミフィケーション環境の開発, 情報処理学会関西支部支部大会講演論文集, vol.2016, (Sept. 2016).
	\bibitem{hikawa}樋川一幸, 松田滉平, 中村聡史: コミュニケーションチャネルに入り込む研究室実験BOTの提案と運用, 情報処理学会研究報告グループウェアとネットワークサービス,vol.3, pp. 1–7(Mar. 2019).
	\bibitem{omura}大村裕, 渡部卓雄: プログラム理解のためのコードリーディング支援ツールの提案と実装, 日本ソフトウェア科学会講演論文集, vol.31, pp.443–446, (Sep. 2014).
	\bibitem{ishio}石尾隆, 田中昌弘, 井上克郎: ソースコード上での情報タグ伝播によるコードリーディング支援, ウィンターワークショップ論文集, Vol.2008, pp.31–32, (Jan. 2008).

	%プログラミングゲーム関連
	\bibitem{topcoder}Topcoder, \url{https://www.topcoder.com/}.
	\bibitem{codegolf}浜地慎一郎: Code Golf, \url{http://shinh.skr.jp/dat_dir/golf_prosym.pdf}.
	\bibitem{seccon}SECCON, \url{https://www.seccon.jp}.
	\bibitem{robocode}Robocode, \url{https://robocode.sourceforge.io/}.

	\bibitem{scratch}Scratch, \url{https://scratch.mit.edu/}.
	\bibitem{viscuit}Viscuit, \url{https://www.viscuit.com/}.
	\bibitem{dolittle}ドリトル, \url{https://dolittle.eplang.jp/}.


	\bibitem{joshua}J.Shi et al: Pyrus: Designing A Collaborative Programming Game to Promote Problem Solving Behaviors, Proceedings of the 2019 CHI Conference on Human Factors in Computing Systems, No. 656, pp. 1–12 (May.2019).
	\bibitem{minakuchi}水口充: 成績評価のためのプログラミングゲームの設計と実践, 研究報告エンタテインメントコンピューティング(EC), vol. 2016, pp. 1–7(July.2016).
	\bibitem{mitani}三谷将大, 寺田実: Webアプリケーションによるゲーミフィケーションを用いたプログラミング上達支援システム, 第27 回インタラクティブシステムとソフトウェアに関するワークショップ, (Sept.2018).
	\bibitem{mashitani}増谷海人, 赤澤紀子: 仮想現実を用いた初学者向けプログラミング学習システムの提案, 2018年度情報処理学会関西支部支部大会講演論文集,vol. 2018, (Sept.2018).


\end{thebibliography}