\documentclass[a4j,12pt]{jarticle}
\usepackage{url}
\usepackage{epsfig}
\usepackage{tabularx}
\usepackage{slashbox}
\usepackage{comment}
\usepackage{subfigure}

\def\subfigcapskip{0pt}

\usepackage{ccaption}
\setlength{\abovecaptionskip}{0pt}
\setlength{\belowcaptionskip}{0pt}
\setlength\floatsep{20pt}
\setlength\abovecaptionskip{10pt}
\setlength\belowcaptionskip{10pt}

\newlength{\leftcolumn}
\setlength{\leftcolumn}{0.40\textwidth}
\newlength{\rightcolumn}
\setlength{\rightcolumn}{0.55\textwidth}

\addtolength{\topmargin}{-.75cm}
\addtolength{\oddsidemargin}{0.18cm}
\addtolength{\textwidth}{2zw}
\addtolength{\textheight}{0.5cm}
\addtolength{\footskip}{-0.8cm}
\setlength{\parindent}{2.5ex}
\newcommand{\normaldefault}[1]{\default{}{#1}{#1}}
\newcommand{\default}[3]{\frac{#1\,:\,#2}{#3}}

%図が入りやすくなる魔法のコマンド
  \setcounter{topnumber}{9} % 頁上部の最大float数
  \setcounter{totalnumber}{9} % 1頁の 〃
  \setcounter{dbltopnumber}{9} % twocolumn時の頁上部の最大float数
  \renewcommand\topfraction{.9} % 頁上部のfloatで占める最大の割合
  \renewcommand\textfraction{.001} % 1頁のテキスト部の占める最小割合
  \renewcommand\dbltopfraction{.9} % twocolumn時の topfraction
  \renewcommand\dblfloatpagefraction{.9} % twocolumn時の floatpagefraction
%魔法

% \graphicspath{{../image/}}
\begin{document}
\mbox{}
%**まえがき**

%**{表紙}**
%\pagestyle{empty}
%\begin{titlepage}
    \fbox{\Large 2021年1月7日提出}
      \vspace*{160truept}
    \begin{center}
      \fbox{\Large 修士論文}

      \vspace{20truept}
      \fbox{
        \begin{tabular}{c}
          {\Huge エンタテインメントを用いた}\\
          {\Huge プログラミング初学者の}\\
          {\Huge 学習意欲促進システムに関する研究} 
        \end{tabular}
      }

      \vspace{110truept}

      \fbox{
        \begin{tabular}{c}
          {\Large 指導教員:西田健志 准教授}\\
          % \vspace{10truept}
          {\Large 副指導教員:大月一弘 教授}
        \end{tabular}
      }
      \vspace{50truept}

      \fbox{
        \begin{tabular}{c}
          {\Large 神戸大学大学院国際文化学研究科}\\
          % \vspace{10truept}
          {\Large グローバル文化専攻情報コミュニケーションコース}\\
          % \vspace{10truept}
          {\Large 学籍番号・氏名 198c125c 岡 大貴}
        \end{tabular}
      } 

  \end{center}
\end{titlepage}

\pagestyle{myheadings}                       % 右上にページ

\newtheorem{theorem}{定理}       % theorem
\newtheorem{definition}{定義}

\renewcommand{\thepage}{\roman{page}\,\,\,}  % 英語ページ
\setcounter{page}{1}
\setlength{\baselineskip}{21pt}
\begin{center}
  {\Large エンタテインメントを用いた\\プログラミング初学者の\\学習意欲促進システムに関する研究}\\
  \vspace{20truept}
    \large{所属専攻・コース:グローバル文化・情報コミュニケーション\\}
    \large{氏名:岡 大貴\\}
    \large{指導教員氏名:西田 健志\\}
\end{center}

\section*{内容梗概}

プログラミングが重要視されている昨今では,プログラミングの授業が小学校に導入されるなど,プログラミングの一般教養化が推し進められている.しかしプログラミングを楽しむためにはある程度の習熟を必要とし,学習中に挫折してしまうプログラミング初学者も少なくない.プログラミングの楽しさを実感させるための初学者向けプログラミング学習コンテンツも存在するが,その多くは小・中学生を対象として設計されておりそれ以上の年代に対しては効果が薄い場合がある.またその多くがビジュアルプログラミング言語を用いているため,実践的なソフトウェア開発で用いるテキストプログラミング言語とは乖離がある場合もある.

そこで本研究ではテキストプログラミング言語を用いつつも,プログラムを書かずとも楽しめる初学者向けの学習・興味喚起を目的とした2つのアプリケーションを開発した.1つ目はエンタテインメントの要素を取り入れることによりコードリーディングを促進するソースコード閲覧システムである.このシステムではクイズ,占いといったエンタテインメントを交えてGitHubにあるソースコードを表示し,ユーザが読解それを読み解くことにより,楽しみながらコードリーディング・プログラムに関するコミュニケーションを促進することを目指した.実装したアプリケーションを使用し運用を行い,その後アンケート調査・ディスカッションを行った結果,クイズに関しては楽しかったという回答が得られたが,ややプログラミング初学者にとっては難解であるという意見が得られた.また表示するソースコードを選択するアルゴリズムの改善や,日常的な使用を促すために更なる工夫が必要であり,これらの問題点を今後改善し,より多くの初学者を対象にワークショップ等を行う予定である.

2つ目はリアルタイム性・アドリブを取り入れた対人形式のプログラミングゲームである.このシステムでは習熟度の高いプログラマ2人がプログラミングスキルによって力量を競って勝負することができ,プログラミング初学者でも観戦して楽しむことができる.このゲームの観戦によってプログラミングへの好奇心を高めるとともに,習熟したプログラマへの憧れを創出することで,初学者のプログラミングに対する興味関心を高めることを目指した.評価実験では実際に2人のプログラマの対戦を初学者に観戦させ,プレイヤであるプログラマと観戦した初学者の双方にアンケート調査を行った.実験結果としてゲームのプレイ・観戦は楽しく,プログラミングに対する興味が高まったなどシステムに対する肯定的な意見が多く得られたが,ゲームシステム・デザインの改善点に関するコメントも多く得られた.またプログラマがプレイ時に記述したプログラムを分析した結果,戦略の多様化・より可読性の高いプログラムの表示などいくつかの改善点が見つかった.

この結果を受け,今後は提案システムを改善すると共に,より多くのプログラミング初学者を含むプログラマにシステムを使用させ,フィードバックを得ることにより,より良いシステムの構築を目指す.

\newpage
\markright{}
\tableofcontents            %  目次
\markright{}
\newpage

%###################### 本文 #######################
\renewcommand{\thepage}{\arabic{page}\,\,\,}  % 数字ページ

\setcounter{page}{1}
\setlength{\baselineskip}{19pt}

\newpage
\section{はじめに}

高速にIT化の進む近年では,スマートフォンの普及を皮切りに,IoT,AI,クラウドコンピューティングといった単語がメディアで散見され,コンピュータが着実に人々の生活を満たしつつあるといえる.コンピュータが遍在する現代において,人とコンピュータの接点は圧倒的に多くなり,コンピュータを専門的に扱う人でなくともそれに触れて暮らすことが当たり前となり,コンピュータとの親和性を高めることが必要である.

近年ではその高速な社会のIT化に伴い,プログラミングが重要性を増している.以前は「プログラミングはエンジニアや理系の学生がやるもの」というような,専門家だけの技能であるという認識が強かったが,プログラミングを英語のように一般教養とし,プログラミングに関わる仕事に就かない人もある程度理解していることが望ましいという風潮がある.国際的にもロシアでは2009年からプログラミング教育が導入され,英国では2014年から「Computing」というコンピュータサイエンス,情報技術,デジタルリテラシーの3分野からなる科目が導入されており\cite{survey},日本においても今年度から小学校でのプログラミング教育が必修化されている\cite{guide}.


プログラミングという行為は機能的であるだけでなく,自身のアイディアをコンピュータを介して表現する手段でもあり,プログラミングスキルを向上させることは自己表現の可能性を広げることにつながる.近年ではProcessingやopenFrameworks,Arduinoなどの登場により,プログラミングにより創造的な表現を生み出す「クリエイティブ・コーディング」やメイカームーブメントが流行し,コンピュータ・サイエンス等の知識がなくともプログラミングによってものづくりをし,発表できる土壌が整いつつある.

しかしプログラミング教育の現場やその学習教材では,プログラミングの機能にのみ焦点が当たりがちで,プログラミングの「楽しさ」が疎かにされていることが少なくない.プログラミングをする上で楽しさを感じることは,学習の観点からも重要である.松本らによるC言語プログラミングの授業では、学習の楽しさや、プログラミングへの興味が高いほど授業の学習率が高いことが示されている\cite{matsumoto}.

しかし最初から理解が容易で楽しさを感じやすい読み書きや運動とは異なり,プログラミングは楽しさを感じるまでにある程度の習熟を要する.よく書かれた小説などの文章やプロスポーツでのファインプレーには初心者でも心動かされるもので,そうした小説家やスポーツ選手への憧れが自ら書くことや体を動かすへの感心を高めるが,プログラミングの場合にはそのような憧れをもつ機会に乏しいのが現状である.

プログラミング初学者にプログラミングの楽しさを実感させるために設計された学習コンテンツは多数存在し,ScratchやViscuitなどのビジュアルプログラミング言語(VPL)がその代表的な例である.これらは実際に学校教育で導入され,小・中学生のプログラミング教育に貢献している.しかしそれらは小・中学生など若い年代をターゲットに設計されているため,高校生や大学生,それ以上の年代にとっては楽しさを感じにくい場合がある.また実践的なソフトウェア開発においてはテキストプログラミング言語(TPL)を用いる場合がほとんどであるため,VPLとTPLの間に乖離があることも問題である.

本研究では従来のシステムがターゲットとしている世代よりも上の世代のプログラミング初学者を対象に,エンタテインメントの要素を用いることでプログラミングに対する抵抗感を減らし,プログラミングを楽しんでもらうためのシステムの開発を目指した.具体的には初学者のコードリーディングを促進するアプリケーション,プログラミングゲームを用いてプログラミングに対する興味喚起を行うアプリケーションの2つを開発し,それぞれ初学者を対象に運用を行うことでその効果を検証した.


本論文では以降,2章で研究指針について述べた後,3章でゲーミフィケーションを用いたコードリーディング支援システムについて述べ,4章でプログラミング初学者の興味喚起を目的としたプログラミングゲームについて述べる.5章で本論文をまとめる.
\markright{}

\newpage
\section{コードリーディング支援システムに関する研究}
本項では,プログラミング初学者向けに作成したエンタテインメントの要素を用いたコードリーディングシステムについて紹介し,その運用結果と課題について議論する.

\subsection{関連研究}

GitHubを活用した研究はいくつかある.

\subsection{設計指針}
システムを実装するにあたり以下の設計指針を設けた.
\begin{enumerate}
  \item 遊び(エンタテインメント)の要素を取り入れる
  コードリーディングはプログラミングにおいて重要なスキルの1つであるが,プログラムを読解する経験の少ないプログラミング初学者にとって,淡白なプログラムをただ読み込むことは難しい.よってエンタテインメントの要素を付与してプログラムを読ませることで,コードリーディングの心理的障壁を下げることを目指した.
  \item 日常的に使用するツールに組み込む
  アスリートが継続的なトレーニングを怠らないように,ある技能を高めるには日常的な鍛錬を要する.プログラミングにおけるコードリーディングでも同様であると考え,極力システム利用のハードルを下げるためにシステムを日常的に使用するデバイス・アプリケーションから使用できるように設計した.
  \item プログラムを読むことに専念させる
  プログラムの記述を促進するならば対象のソースコードを1つの言語に絞るべきという考え方もある.コードを書くことが目的であれば,複数の言語を同時進行させることは混乱を招き,デメリットが大きい.しかし,コードを読むことが目的であれば,上級者との会話のきっかけができる,将来的に適材適所で言語を使い分けることにつながる,飽きずにシステムを使用させ,コードの読解に興味を持たせることにつながるなどメリットが大きい,従って本システムでは,多様な言語に触れられる機能と1つの言語に読解を絞った機能を実装した.
\end{enumerate}

\subsection{提案システム}
\subsubsection{実装機能}
システムで使用できる2つの機能について説明する.
\begin{enumerate}
  \item クイズ機能
  この機能はユーザが使用した際にGitHubから取得したランダムなプログラムを表示する.ユーザは表示されたプログラムを読み,どのプログラミング言語で記述されたプログラムかを回答する.正解した場合は得点が得られ,取得した累計得点の多さによるランキングが表示される.
  \item 占い機能
  この機能をユーザが使用するとGitHubからランダムに得られたプログラムが表示され,ユーザはそれを読解し自分なりの解釈をすることで,その日の運勢を占う.表示されたソースコードの意味を解釈することが必要であるため,必然的にコードの読解が必要となる.この機能においては表示されるプログラムをJavaScriptで記述されたもののみにした.
\end{enumerate}
\subsection{プロトタイプシステムとケーススタディ}
初めに,提案システムをチームコミュニケーションツールであるSlackのbotとして実装した.システムを利用する際は,このbotを導入しているチャンネルでコマンドを入力することで各機能を使用できる.
このシステムを筆者が所属する研究室のSlackに導入し,研究室のメンバーを対象としたケーススタディを行った..研究室のメンバーは3名のプログラミング初学者であり,1週間自由に機能を使用してもらい,各コマンドの使用回数を調べた.またアンケートによって所感を調査した.

\subsection{課題と考察}



\newpage
\section{競技性・観戦性を拡張したプログラミングゲームの開発}
本稿では,プログラミング初学者の興味喚起を目的としたプログラミングゲームのプロトタイプについて紹介し,設計指針に基づくデザインの詳細や行った評価実験と見つかった課題,実験に関する考察について議論する.

\subsection{関連研究・関連システム}

\subsubsection{プログラミングを用いたエンタテインメントシステム}
プログラミングとエンタテインメントを掛け合わせたコンテンツはいくつか存在する.TopCoder\cite{topcoder}などの競技プログラミングやコードゴルフ\cite{codegolf},SECCON\cite{seccon}などのハッキングコンテストが有名であり,プログラマの間でも根強い人気がある.またプログラミングゲームとしてはRobocode\cite{robocode}が有名である.しかしこれらはある程度プログラミングに習熟したプログラマ向けのコンテンツであるため,数学やセキュリティ,コンピュータサイエンス等の知識を要するためプログラミング初学者が参加するにはハードルが高い.

\subsubsection{エンタテインメントを活用したプログラミング学習支援システム}
エンタテインメントの要素を盛り込むことでプログラミング学習を促進しようとした研究は多くある.Joshaらはプログラミングゲームを用いてプログラミング初学者の問題解決能力を向上させるためのシステムを作成している\cite{joshua}.また水口の研究ではプログラミングの講義における成績評価にロボットバトルシミュレーション型のプログラミングゲームを活用している\cite{minakuchi}.三谷らの研究ではキャラクタをプログラムで制御するプログラミングゲームでプログラミングスキルの向上を図っている\cite{mitani}.なお増谷らの開発したVLogic\cite{mashitani}ではVR空間上にブロックベースのプログラミングゲームを実装することで手足を使ってプログラミングを体験することができ,プログラミングに対する興味喚起を行っている.これらはプログラミングの習熟度が高くなくても使用できるが,従来のプログラミングゲーム同様静的なゲーム展開であり,プログラマ同士のリアルタイムな駆け引きやアドリブといった観戦を楽しむ設計は成されていない.

\subsection{設計指針}

提案システムの実装にあたり,初学者の興味関心を高めるために3つの設計指針を設けた.

\begin{enumerate}
	\item 観戦するにあたり,高度な専門知識を必要としない
	
	初学者が提案システムでの対戦を観戦するにあたり高度な専門知識を必要としてしまっては,利用の心理的ハードルを上げかねない.極力前提知識なしに理解し,楽しめるようにデザインする必要がある.
	
	\item 手間がかからない

	初学者にとってプログラミング学習をする際に環境構築や普段使用しない独自ソフトウェアのインストールはモチベーションを下げかねない.今回は提案システムをJavaScriptによって制御可能なWebアプリケーションとして実装し,初学者が実際にプログラミングを行わずとも見るだけで利用できるようにした.
	
	\item リアルタイムな駆け引き・アドリブを取り入れる

	従来のプログラミング教育コンテンツとは異なり,スポーツなどの観戦する競技で見られるようなリアルタイムな駆け引き・アドリブの要素を提案システムに組み込むことで観戦して楽しめるようなデザインをする.
\end{enumerate}

\subsection{提案システム}

本研究で提案するシステムでは,プログラミング初学者の興味喚起をするためにいくつかの工夫を施した.今回実装したシステムは,プログラマ同士がリアルタイムにプログラミングを行うことでキャラクタを制御し,対戦するという対人形式のプログラミングゲームである.この対戦の様子を初学者に観戦させることで,初学者のプログラミングに対する興味を高める.プログラミングゲームとして実装した理由としては,キャラクタがプログラムによって動作するというゲームの視覚的な出力が,初学者にとってプログラムの出力を理解する助けになると考えたためである.ゲームジャンルとしては,シューティングゲームの体裁をとった.これはシューティングゲームが「敵の攻撃を避けて,敵を攻撃する」というプリミティブなゲームシステムであり,見ていて展開を理解しやすいと考えたためである.またゲームUIはLivecodeLabやHydraなどのビジュアルライブコーディング環境を踏襲し,ゲーム画面上にエディタを重畳することで,プログラムとその出力の双方を同時に見ることが可能なように設計した.またゲームは以下の3つのフェイズに分かれている.

\subsubsection{自機プログラミングフェイズ}
対戦が開始するとこのフェイズに移行する.ここでは予めエディタにプレイヤが操作するキャラクタのコンストラクタが記述されており,プレイヤはパラメータを書き換えることができる.具体的なパラメータにはappearance,life,clock,powerがある.appearanceはキャラクタの外見であり,文字列を指定できるため,絵文字などを使ってプレイヤの好きな見た目を選ぶことが可能である.lifeはキャラクタの体力であり,いわゆるHP(ヒットポイント)を表している.非負の整数を指定でき,この値が0以下になるとプレイヤはゲームに敗北する.clockはキャラクタが行動できる回数の多さを表しており,非負の整数を指定できる.プレイヤは後述する行動プログラミングフェイズにおいて自分のキャラクタを制御するプログラムを記述し対戦するが,その際に記述したプログラムは10秒間ループして実行される.このループのインターバルを決めるのがclockであり,値が大きいほどインターバルは短くなる.powerはキャラクタの攻撃力を表しており,これも非負の整数を指定できる.この値が大きいほど,自分が操作するキャラクタの攻撃が相手キャラクタに命中した際に削るlifeの値が大きくなる.双方のプレイヤが各パラメータを記述し終わると次の行動プログラミングフェイズに移行する.

\subsubsection{行動プログラミングフェイズ}
このフェイズに進むと,自機プログラミングフェイズで記述したコンストラクタを元に各プレイヤが操作するキャラクタのインスタンスが作成され,ゲーム画面が表示される.各プレイヤはプログラムをエディタに記述し,作成したキャラクタを操作する.プレイヤは条件分岐や繰り返しなど従来のJavaScriptの文法の他に独自に用意されたプロトタイプメソッドを使うことができる.用意したメソッドにはキャラクタを移動するメソッド(moveUp(), moveDown(),randomMove())とキャラクタが攻撃を行うメソッド(shot())などがある.またプログラム内で各キャラクタのパラメータを参照することもできる.両プレイヤがプログラムを記述し終わると次のゲームフェイズに移行する.なおこのフェイズでは相手プレイヤがどのようなプログラムを記述しているかは見ることができない.

\subsubsection{ゲームフェイズ}
このフェイズでは行動プログラミングフェイズで記述したプログラムが10秒間ループして実行され,ゲームが進行する.この段階で両プレイヤは相手プレイヤが記述したプログラムを閲覧することができる.このフェイズにおいて相手キャラクタを攻撃し,lifeの値を0いかにしたプレイヤの勝利となる.勝敗が決まらない場合はプログラム終了時の各パラメータを引き継いだまま行動プログラミングフェイズに戻り,再度プログラミングしゲームフェイズに移行するという過程を勝敗が決まるまで繰り返す.

\subsection{評価実験}
提案システムの使用・観戦に関する感想や影響,用法を調査するために評価実験を実施した.システムをプレイするプログラマと観戦するプログラミング初学者を集め,システムでの対戦と観戦を実施し,アンケート調査と実際に対戦で使用されたプログラムのログを分析することでシステムを評価した.

\subsubsection{実験参加者}
実際にゲームをプレイするプログラマとしては,プログラミング(主にオブジェクト指向言語)の経験が3年以上ある大学院生2名(男性)に声をかけた.2名とも日常的にアクションやシューティング等のジャンルのゲームをプレイするため,システムをプレイする際にゲームに不慣れなためハンデが生まれることはないと思われる.また両者とも3年以上JavaScriptを使用した経験がある.
またプレイを観戦するプログラミング初学者は,国際人間科学部にて開講されていた講義「プログラミング基礎演習1」の受講者をとした.受講者のうち,実験に参加した者は77名であり,うち37名が男性,40名が女性であった.またこの講義ではJavaScriptにおける変数,条件分岐,繰り返しなどの基礎的な文法を教えており,実験参加者は実験を行う時点でこれらを学習済であった.なお,うち23名は授業以前にプログラミングを学習した経験があったが,ソフトウェア開発を行ったことのある者はいなかった.
またプログラマ含む実験参加者の全員が,競技プログラミングやプログラミングゲームなどのプログラミングを題材としたエンタテインメントを使用した経験がなかった.

\subsubsection{実験内容}

初めにプログラマ2人に簡単な事前アンケートを行った後,提案システムにある程度慣れ,用法を理解してもらう必要があるため,システムの練習をするための期間を設けた.システムの使用方法,ゲームシステム,独自に用意したメソッドなどについて説明した後,11/13~11/19の1週間システムを自由に使用させた.また,ただシステムを使用させただけではゲームに対する理解度が上がらない可能性があるため,期間中に2つのタスクをこなさせた.1つは1人以上とシステムを使った対人戦を行うことであり,もう1つは相手がランダムな戦略を実行してくる対CPU戦において,勝率が高いと考えられるプログラムを作成することである.なお期間中はシステムに関する意見・疑問を逐次報告させ,システム使用における問題を改善した.
練習期間が終わった翌日に,プログラマ同士の対戦を行った.対戦はシステムに関するプログラマ同士のコミュニケーション等の所作を観察するため,感染症対策を徹底した環境で対面にて行った.両者の対戦時の画面を録画し,対戦後にシステムに関する事後アンケートを行った.なお練習期間中・対戦中にプログラマが記述した全てのコードのログを収集した.
そして「プログラミング基礎演習1」の最終講義で受講者に事前アンケートを行った後,初学者が観戦しやすいように対戦動画を編集したものをzoomを介して閲覧させた.またその後にシステムに関する事後アンケートを行った.

\subsubsection{実験結果}

\begin{enumerate}
	\item アンケート結果

	\item コードログ分析結果
\end{enumerate}

\subsubsection{考察}
評価実験に関する考察について述べる.プログラマ同士の対戦においては,行動プログラミングフェイズで設定したappearanceについて「かわいい」などとコメントしていた.また対戦後にお互いのプログラムの内容や今までのプログラミング経験に関するコミュニケーションをとっており,システムを使用することでプログラマ同士のコミュニケーションを促進できていたと考えられる.
初学者の対戦の観戦においては,ライブでプログラマ同士が対戦している状況を用意することが困難であったため今回は動画を閲覧するという状況を用意したが,動画を見ただけでは実際に人が対戦しているという感覚が希薄であり,「プログラマがプログラミングしている」場面を見せるためには更なる工夫が必要であると感じた.
また今回の評価実験において,システムに対する肯定的な評価が得られたものの,どのような要素が初学者のプログラミングに対する興味に影響を与えていたのか,特に提案システムにおいて独特な要素であるリアルタイムな駆け引き・アドリブを誘発する要素の影響について調査する必要があると考えられる.

\subsection{課題}







\newpage
\section{まとめ}

% 本論文では,プロジェクションマッピングによる情報提示を行うウェアラブルデバイスにより,紙の書籍による読書体験を拡張するシステムを提案した.提案システムはウェブカメラから取得した紙面の画像を取得し,光学文字認識により紙面の文章データを得る.そして取得した文章を形態素解析により形態素に分け,名詞のみを抽出し,紙面に表示する.ユーザが紙面に表示された名詞を選択すると,その名詞の意味または関連画像が検索され,紙面上に表示される.このシステムを用いて読書することにより,読者は分からない名詞が出てきた際にも,読書を中断して辞書や電子書籍で調べることなく調べ物が可能で,より拡張された読書を体験できると考えられる.
% 提案システムが,実際に読書の利便性を高められているか,実装した機能について調査した.その結果,デバイスの機能に関しては便利だという意見が得られたが,文字・画像の表示位置,トラッキングの精度と入力インタフェースに改善すべき点が見つかった.
% 今後の課題として,トラッキング精度を改善し,より適切な位置へ情報を投影することが必要である.入力インタフェースとして用いたLeapMotionが使いづらいという意見が多く寄せられたため,読書行為を阻害しないインタフェースの検討が必要である.なおデバイスの情報処理に時間がかかり,文字・画像の投影タイミングが遅いという結果も得られたため,プログラム内容の簡略化やより高速な言語を使用してのシステム開発が必要だと考えられる.


\newpage
\addcontentsline{toc}{section}{\protect\numberline{}{謝辞}}
\markright{}
\include{5_thanks}

\newpage
\addcontentsline{toc}{section}{\protect\numberline{}{参考文献}}
\markright{}
\begin{thebibliography}{99}
	
	%thesis
	\bibitem{takano} 
	 高野健太郎, 大村賢悟, 柴田博仁: ``短編小説の読みにおける紙の書籍と電子書籍端末の比較,'' 情報処理学会第141回HCI研究会, Vol. 141, No. 4, pp.1--8 (Jan. 2011).
	
	\bibitem{shibata} 
	柴田博仁, 大村賢悟: ``答えを探す読みにおける紙の書籍と電子書籍端末の比較,'' 情報処理学会第141回HCI研究会, Vol. 141, No. 5, pp. 1--8 (Jan. 2011).
	
	 \bibitem{pablo}
	 P. Delgado, C. Vargas, R. Ackerman, and L. Salmer\'on: ``Don't Throw Away Your Printed Books: A Meta-analysis on the Effects of Reading Media on Reading Comprehension,'' {\it Journal of Educational Research Review}, Vol. 25, pp. 23--38 (Nov. 2018).
	 
	 \bibitem{inagawa1}
	稲川暢浩, 藤波香織: ``読書体験を豊かにするための本の拡張に関する研究,'' 情報処理学会第70回全国大会, pp. 131--132 (Mar. 2008).
	 
	 \bibitem{inagawa2}
	稲川暢浩, 藤波香織: ``読書中に低い妨害感で効果的に付加情報を伝達するための情報提示方法,'' 情報処理学会第135回HCI研究会, Vol. 2009, No. 12, pp. 1--8 (Dec. 2009).
	 
	 \bibitem{hashimoto}
	 橋本直己, 櫻井淳一: ``インタラクティブなプロジェクションマッピングの実現,'' 映像情報メディア学会誌, Vol. 67, No. 2, pp. 60--63 (Jan. 2013).
	
	\bibitem{isogaki}
	磯垣広野, K. S. Kyung, 澤野知佳, 福井 悠, 小林孝浩, 鈴木宣也, 関口敦仁: ``可動鏡を使ったマルチプロジェクションシステム -Ptolemy-,'' 情報処理学会シンポジウム論文集, Vol. 2007, No. 4, pp. 85--86 (Mar. 2008).
	
	\bibitem{doi}
	土井麻由佳, 宮下芳明: ``プロジェクションマッピングによる箏への演奏提示,'' 第23回インタラクティブシステムとソフトウェアに関するワークショップ論文集(WISS 2015), pp. 181--182 (Feb. 2015).
	
	
	\bibitem{gupta}
	S. Gupta and C. Jaynes: ``The Universal Media Book: Tracking and Augmenting Moving Surfaces with Projected Information,'' {\it Proc. of the ACM International Symposium on Mixed and Augmented Reality (ISMAR 2006)}, pp. 177--180 (Oct. 2006).
	
	\bibitem{suzuki}
	鈴木若菜, 竹田一貴, 外山託海, 黄瀬浩一: ``プロジェクタを用いた情報投影による印刷文書へのインタラクティブ性の付加,'' 電子情報通信学会技術研究報告, Vol. 111, No. 317, pp. 69--74 (Nov. 2011).
	
	\bibitem{ota}
	太田脩平, 寺田 努, 塚本昌彦: ``装着型プロジェクタと可動鏡による周辺状況を考慮した映像投影手法,'' エンタテインメントコンピューティングシンポジウム (EC 2012) 論文集, Vol. 2012, No. 1,  pp.1--7 (Mar. 2012).
	
	\bibitem{pranav}
	P. Mistry, P. Maes, and L. Chang: ``WUW - Wear Ur World - A Wearable Gestural Interface,'' {\it Proc. of the ACM International Conference on Human Factors in Computing Systems Extended Abstracts (CHI EA 2010)}, pp. 4111--4116 (Apr. 2009).
	
	\bibitem{chris}
	C. Harrison, H. Benko, and A. D. Wilson: ``OmniTouch: Wearable Multitouch Interaction Everywhere,'' {\it Proc. of the ACM symposium on User interface software and technology (UIST '11)}, pp. 441--450 (Oct. 2011).
	
	\bibitem{christian}
	C. Winkler, M. Broscheit, and E. Rukzio: ``NaviBeam: Indoor Assistance and Navigation for Shopping Malls through Projector Phones,'' {\it Proc. of the ACM Conference on Human Factors in Computing Systems (CHI 2011)}, (Jan. 2011).
	
	
	 \bibitem{karitsuka}
	  狩塚俊和, 佐藤宏介: ``プロジェクタ投影型ウェアラブル複合現実感システム, 情報処理学会研究報告コンピュータビジョンとイメージメディア (CVIM), Vol. 104, pp. 141--146 (Sep. 2003).
	  
	\bibitem{leap}LeapMotion, \url{https://www.leapmotion.com/ja/}.
	
	\bibitem{opencv}OpenCV, \url{https://opencv.org/}.
	
	\bibitem{google}Google Cloud Vision API, \url{https://cloud.google.com/vision/}.
	
	\bibitem{wikipedia}Wikipedia, \url{https://ja.wikipedia.org/wiki/}.
			
	\bibitem{mecab} 
	MeCab: MeCab: Yet Another Part-of-Speech and Morphological Analyzer,  \url{http://taku910.github.io/mecab/}.
	
	\bibitem{morimi}森見登美彦: 夜は短し歩けよ乙女, 角川書店 (2006).

	
\end{thebibliography}


\end{document}