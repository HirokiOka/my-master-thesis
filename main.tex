\documentclass[a4j,12pt]{jarticle}
\usepackage{url}
\usepackage{epsfig}
\usepackage{tabularx}
\usepackage{slashbox}
\usepackage{comment}
\usepackage{subfigure}

\def\subfigcapskip{0pt}

\usepackage{ccaption}
\setlength{\abovecaptionskip}{0pt}
\setlength{\belowcaptionskip}{0pt}
\setlength\floatsep{20pt}
\setlength\abovecaptionskip{10pt}
\setlength\belowcaptionskip{10pt}

\newlength{\leftcolumn}
\setlength{\leftcolumn}{0.40\textwidth}
\newlength{\rightcolumn}
\setlength{\rightcolumn}{0.55\textwidth}

\addtolength{\topmargin}{-.75cm}
\addtolength{\oddsidemargin}{0.18cm}
\addtolength{\textwidth}{2zw}
\addtolength{\textheight}{0.5cm}
\addtolength{\footskip}{-0.8cm}
\setlength{\parindent}{2.5ex}
\newcommand{\normaldefault}[1]{\default{}{#1}{#1}}
\newcommand{\default}[3]{\frac{#1\,:\,#2}{#3}}

%図が入りやすくなる魔法のコマンド
  \setcounter{topnumber}{9} % 頁上部の最大float数
  \setcounter{totalnumber}{9} % 1頁の 〃
  \setcounter{dbltopnumber}{9} % twocolumn時の頁上部の最大float数
  \renewcommand\topfraction{.9} % 頁上部のfloatで占める最大の割合
  \renewcommand\textfraction{.001} % 1頁のテキスト部の占める最小割合
  \renewcommand\dbltopfraction{.9} % twocolumn時の topfraction
  \renewcommand\dblfloatpagefraction{.9} % twocolumn時の floatpagefraction
%魔法

\begin{document}
\mbox{}
%**まえがき**

%**{表紙}**
%\pagestyle{empty}
%\begin{titlepage}
    \fbox{\Large 2021年1月7日提出}
      \vspace*{160truept}
    \begin{center}
      \fbox{\Large 修士論文}

      \vspace{20truept}
      \fbox{
        \begin{tabular}{c}
          {\Huge エンタテインメントを用いた}\\
          {\Huge プログラミング初学者の}\\
          {\Huge 学習意欲促進システムに関する研究} 
        \end{tabular}
      }

      \vspace{110truept}

      \fbox{
        \begin{tabular}{c}
          {\Large 指導教員:西田健志 准教授}\\
          \vspace{10truept}
          {\Large 副指導教員:大月一弘 教授}
        \end{tabular}
      }
      \vspace{50truept}

      \fbox{
        \begin{tabular}{c}
          {\Large 神戸大学大学院国際文化学研究科}\\
          \vspace{10truept}
          {\Large グローバル文化専攻情報コミュニケーションコース}\\
          \vspace{10truept}
          {\Large 学籍番号・氏名 198c125c 岡 大貴}
        \end{tabular}
      } 

  \end{center}
\end{titlepage}

\pagestyle{myheadings}                       % 右上にページ

\newtheorem{theorem}{定理}       % theorem
\newtheorem{definition}{定義}

\renewcommand{\thepage}{\roman{page}\,\,\,}  % 英語ページ
\setcounter{page}{1}
\setlength{\baselineskip}{21pt}
\begin{center}
  {\Large エンタテインメントを用いた\\プログラミング初学者の\\学習意欲促進システムに関する研究}\\
  \vspace{20truept}
    所属専攻・コース:グローバル文化・情報コミュニケーション\\
    氏名:岡 大貴\\
    指導教員氏名:西田 健志\\
\end{center}

\section*{内容梗概}

プログラミングが重要視されている昨今では,プログラミングの授業が小学校に導入されるなど,プログラミングの一般教養化が推し進められている.しかしプログラミングを楽しむためにはある程度の習熟を必要とし,学習中に挫折してしまうプログラミング初学者も少なくない.プログラミングの楽しさを実感させるための初学者向けプログラミング学習コンテンツも存在するが,その多くは小・中学生を対象として設計されておりそれ以上の年代に対しては効果が薄い場合がある.またビジュアルプログラミング言語を用いているため,実践的なテキストプログラミング言語とは乖離がある場合もある.


そこで本研究では初学者のテキストプログラミング言語を用いた学習・興味喚起を促すため,2つのアプリケーションを開発した.1つ目はエンタテインメントの要素を取り入れることによりコードリーディングを促進するソースコード閲覧システムである.このシステムではクイズ,占いといったエンタテインメントを交えてGitHubにあるソースコードを表示し,ユーザが読解それを読み解くことにより,楽しみながらコードリーディングを促進することを目指した.実装したアプリケーションを使用し運用を行い,その後アンケート調査・ディスカッションを行った結果,クイズに関しては楽しかったという回答が得られたが,ややプログラミング初学者にとっては難解であるという意見が得られた.また表示するソースコードを選択するアルゴリズムの改善や,日常的な使用を促すために更なる工夫が必要であり,これらの問題点を今後改善し,より多くの初学者を対象にワークショップ等を行う予定である.


2つ目はリアルタイム性・アドリブを取り入れた対人形式のプログラミングゲームである.このシステムでは習熟度の高いプログラマ2人がプログラミングスキルによって力量を競って勝負することができ,プログラミング初学者でも観戦して楽しむことができる.このゲームの観戦によってプログラミングへの好奇心を高めるとともに,習熟したプログラマへの憧れを創出することで,初学者のプログラミングに対する興味関心を高めることを目指した.評価実験では実際に2人のプログラマの対戦を初学者に観戦させ,プレイヤであるプログラマと観戦した初学者の双方にアンケート調査を行った.実験結果としてゲームのプレイ・観戦は楽しく,プログラミングに対する興味が高まったなどシステムに対する肯定的な意見が多く得られたが,ゲームシステム・デザインの改善点に関するコメントも多く得られた.またプログラマがプレイ時に記述したプログラムを分析した結果,戦略の多様化・より可読性の高いプログラムの表示などいくつかの改善点が見つかった.

この結果を受け,今後は提案システムを改善すると共に,より多くのプログラミング初学者を含むプログラマにシステムを使用させ,より良いシステムの構築を目指す.

\newpage
\markright{}
\tableofcontents            %  目次
\markright{}
\newpage

%###################### 本文 #######################
\renewcommand{\thepage}{\arabic{page}\,\,\,}  % 数字ページ

\setcounter{page}{1}
\setlength{\baselineskip}{19pt}

\newpage
\section{はじめに}


昨今,自然言語での読み書きや健康のために行う運動のようにプログラミングを一般教養にしようとする動きがある.小学校におけるプログラミング教育の必修化がその最たる例で,将来プログラミングに関わる職業につかない人でもプログラミングに関する知識やその考え方をある程度身につける方が望ましいと考えられている.

しかし最初から理解が容易で楽しさを感じやすい読み書きや運動とは異なり,プログラミングは楽しさを感じるまでにある程度の習熟を要する.よく書かれた小説などの文章やプロスポーツでのファインプレーには初心者でも心動かされるもので,そうした小説家やスポーツ選手への憧れが自ら書くことや体を動かすへの感心を高めるが,プログラミングの場合にはそのような憧れを持つ機会に乏しいのが現状である.
プログラミング初学者にプログラミングの楽しさを実感させるために設計された学習コンテンツは多数存在するが,その多くは小・中学生など若い年代をターゲットに設計されているため,高校生や大学生などには向かない.

本研究ではプログラミング初学者のプログラミングに対する興味関心を高めるため.ゲーミフィケーションやゲームといったエンタテインメントの要素を組み込んだコンテンツを開発し,初学者を対象に運用を行うことでその効果を検証した.

本論文では以降,2章でゲーミフィケーションを用いたコードリーディング支援システムについて述べ,3章でプログラミング初学者の興味喚起を目的としたプログラミングゲームについて述べる.4章で本論文をまとめる.
\markright{}

\newpage
\section{関連研究}

\subsection{読書に関する研究}

読書をテーマにした研究は今までに数多く行われてきたが,その中でも紙の書籍における読書と電子書籍における読書を比較する研究は多く行われている.
柴田らの研究では,業務内容などが記載されたマニュアルから答えを探し出す読みを紙の書籍と電子書籍端末において比較している\cite{shibata}.
この研究では,テキストマニュアルから答えを探す読みを行った際に,紙の書籍を用いた方が60.2\%速く答えを得られるという実験結果を得ている.
さらにDelgadoらのメタ分析においては,ディスプレイにおける文章と印刷された文章を比較した結果,デジタルディスプレイの方が文章の理解度が低くなると結論づけられている\cite{pablo}.
また高野らの研究では,電子書籍と紙の書籍で小説を読書した際に,主観的にどちらの媒体が好みであるかを被験者に評価させ,紙の書籍が最も評価が高かったという結果を得ている\cite{takano}.
これらの研究結果から,電子書籍は今後発展し利便性を増す可能性を秘めつつも,紙の書籍における読書には多くのメリットがあり,主観的にも好まれていることから今後も紙の書籍に対する需要は絶えないと予測できる.

また読書をテーマとする研究において,紙の書籍の読書を拡張するという目的のものも行われている.
稲川らは,既存の読書に別の情報を付加し,従来の読書を拡張させる研究を行っている\cite{inagawa1}.この研究では,センサを用いて紙の書籍のページを取得し,ページに応じた音声や映像を流すという手法で読書体験の拡張を図っている.
翌年には,読書中の情報提示の方法に関する評価も行っている\cite{inagawa2}.
この研究では,紙の書籍の読書中に,読んでいるページに関連する画像を読者に表示し,表示する媒体として据置型ディスプレイに表示する場合とHMDに表示する場合を比較しており,画像を表示するタイミングについても,自動的に表示される手法と読者がページをめくる時に表示する半自動的な手法とを比較している.

これらの研究では,視覚あるいは聴覚を用いた情報付与により読書を拡張しているが,読む本のデータとそれに応じた音声,映像データを予め用意しておく必要がある.本研究では,既存のあらゆる書籍に対応して読書を拡張することを目標としているため,これらの研究では本研究の研究目的を実現できない.


\subsection{固定プロジェクタを用いた研究}

近年,プロジェクタを安価に入手できるようになったことや,その急速な発展からプロジェクタを用いた研究は盛んであり,メディアアートやインタラクティブアート,MR,HCIの分野などで主に使用されている.
橋本らの研究では使用者が壁に対してボールを投げる,あるいは投げる動作をとった際に,その動作に応じて壁が崩れるCGを壁に重畳するインタラクティブなプロジェクションマッピングを行っている\cite{hashimoto}.
また磯垣らは1台のプロジェクタと48枚の可動鏡を用いたインタラクティブアート作品Ptolemyを製作している\cite{isogaki}.この作品は,マウスと画像認識等の入力インタフェースにより,可動鏡と投影コンテンツを操作し,1台のプロジェクタによって複数の位置にコンテンツを投影することのできるものである.
土井らはプロジェクションマッピングによる情報提示を用いて,ユーザを支援するデバイスを提案している\cite{doi}.この研究では,箏の弦にプロジェクションマッピングをすることで,その奏法や演奏タイミングの支援を行っている.
また本型の投影面へ情報投影を行っているGuptaらの研究がある\cite{gupta}.この研究では,大型の白紙の本を投影面とし,カメラとプロジェクタを用いたデバイスにより,そのページのめくり具合から,投影するコンテンツを変化させている.

これらの研究から,プロジェクタを用いることで現実の拡張や既存のサービスの利便性を向上させられると考えられる.しかし,大型の固定プロジェクタを用いるシステムは,使用できる状況が限定的であると考えられ,本研究で提案するシステムのような使用場所を選ばないウェアラブルデバイスとしての実現には,取り入れることのできる要素はあるが,そのまま利用するのは難しい.

\subsection{モバイルプロジェクタを用いた研究}

近年のプロジェクタの小型化に伴い,モバイルプロジェクタを用いた研究が多く行われている.
鈴木らは,モバイルプロジェクタとウェブカメラ,PCから成るデバイスを開発している\cite{suzuki}.
この研究では,データベースに格納された既知の紙の文章に対し,プロジェクタから情報を投影することにより,電子文書のようなサービスを実現している.
太田らは,装着型プロジェクタと可動鏡を用いることにより,周辺の状況により投影面を変更することのできる映像投影法を提案している\cite{ota}.
また,プロジェクタをウェアラブルデバイスとして用いた一例として,MistryらのWUWやHarrisonらのOmniTouchがある\cite{pranav,chris}.
これらはカメラとモバイルプロジェクタからなるデバイスであり,周辺の壁や手などに情報投影を行い,カメラによって手のジェスチャを認識することでデバイスへの入力を可能としたウェアラブルデバイスである.
Winklerらは,プロジェクタフォンというプロジェクタとしての機能を実装したスマートフォンを用いている\cite{christian}.この研究で提案されているのは,ショッピングモールでプロジェクタフォンを腰に装着し床にユーザが行きたい店へのナビゲーションを表示することで,屋内での活動を補助するデバイスである.
狩塚らが行った研究では,モバイルプロジェクタを用いて複合現実感を実現するウェアラブルデバイスが提案されている\cite{karitsuka}.いずれのデバイスもモバイルプロジェクタを用いて,従来のサービスを拡張するデバイスであるが,本研究で提案する読書を拡張するというシチュエーションを想定した研究は筆者の知る限りでは存在しない.



\newpage
\section{競技性・観戦性を拡張したプログラミングゲームの開発}


\subsection{関連研究}

\subsection{設計指針}

\begin{itemize}
	\item ユーザがシステムの利便性を感じる
	\item ユーザの任意のタイミングで情報を取得できる
	\item ユーザの読書を妨害しない
\end{itemize}


\subsection{提案システム}

\subsection{評価実験}

\subsection{課題と考察}






\newpage
\section{実装}


3章で述べたシステム設計をもとにして,プロジェクションマッピングによる情報提示を行うプロトタイプデバイスを実装した.デバイスの外観を図に示す.デバイスの重さは約530gであり,このデバイスを読書中に首からかけて使用することで,機能を使用する.デバイスを使用している様子を図に示す.なおプロジェクタ・ウェブカメラの部分の角度は図のように可変となっており,読書時の姿勢に応じて手動で調整する.ウェブカメラにはMicrosoft社のLifeCam Cinema,プロジェクタにはTenker社のDLP mini projectorを用いた.プロジェクタの投射距離は18〜250cmであり,最大100インチでの投影が可能である.また本デバイスは本から直線距離で約30cm程離して利用することを想定しており,その際のプロジェクタの照射可能範囲は縦約22cm,横約14cmである.同様にカメラのカバー範囲は縦約21cm,横約33cmである.これを図に示す.またデバイスの照射可能距離を図に示す.ウェブカメラ,ハンドジェスチャ認識デバイスはそれぞれPCにUSB接続されている.ハンドジェスチャ認識デバイスのジェスチャをカバーする範囲は,本体を中心に縦約50cm,横約45cmである.なおPCはApple社のMacBook Pro (CPU: Core i5,メモリ: 8GB)を用いた.ソフトウェアに関しては,OpenCVやCloud Vision API,MeCabなどの処理をpythonで行い,プロジェクションマッピングによる描画の処理にはProcessingを用いた.OpenCVとは,オープンソースのコンピュータビジョンライブラリであり,Cloud Vision APIとは,機械学習モデルを用いて画像の内容を取得する画像分析APIである.


\subsection{紙面のトラッキング}
紙面のトラッキングの際には,OpenCVを用いたトラッキングを行った\cite{opencv}.OpenCVで使用できるトラッキングアルゴリズムには,Boosting,MIL,TLD,MedianFlow,KCFの5つが存在するが,リアルタイムで精度の高い認識が求められるため,今回はこの中でも高精度かつ高速に物体を学習し追跡できるKCFを用いた.

\subsection{システムへの入力}

本システムへユーザが入力を行う際は,LeapMotion\cite{leap}によって手の動作を認識することによって行った.LeapMotionとは手のジェスチャを認識することで,コンピュータへの入力を可能とする入力機器である.このデバイスでは,手のひらの中央および各指,各関節の始点と終点を3次元の座標と方向ベクトルで取得できる.またcircle,swipe,key tap,screen tapの4種のジェスチャを認識できる.各ジェスチャを図に示す.本研究では,人差し指の3次元座標とswipe,key tapジェスチャを用いる.ユーザが必要な情報を得たい場合は,key tapジェスチャを認識することにより機能を実行できる.また表示されている名詞を選択する際は,ページの遷移をswipeジェスチャで行い,指定の名詞を人差し指の3次元座標とkey tapジェスチャによって選択する.また検索結果を削除する際もkey tapジェスチャによって行った.


\subsection{文字認識}

読書中のページの文章を取得する.デバイスのウェブカメラから紙面を撮像し,LeapMotionにより指のkey tapジェスチャを認識した際にその時点でのカメラが撮像している画像を取得する.この取得した画像をPCで処理することにより,文字認識を行った.取得した画像には読書中の本の紙面が含まれているため,その画像から光学文字認識により文章を取得する.この際にGoogleのCloud Vision API\cite{google}を使用した.このAPIを用いて検出できた文章に対し,次項で述べる形態素解析を行った.

\subsection{形態素解析}

文字認識によって取得した文章を日本語形態素解析機MeCab\cite{mecab}によって形態素解析した.
形態素解析とは,自然言語によって記述された文章を,対象言語の文法や辞書の情報に基づき,文章の最小単位である形態素に分割し,それぞれの形態素の分析を行うことである.
形態素に分けた文章から名詞のみを抽出し,プロジェクタにより紙面に表示する.この際に,紙面の文章と表示した名詞とが重なり可読性が下がることを避けるため,紙面上部の余白領域に名詞を表示するようにした.



\newpage
\section{評価実験}

本研究では,プロジェクションマッピングにより読者の必要な情報を紙面に表示することで読書体験を拡張するシステムを提案した.このシステムを使用する際,ユーザが便利だと感じるか,適切に情報付与を行えているかを評価することが必要である.本章では,提案システムを実装したデバイスを用い,被験者に実際に紙の書籍で読書をさせることでその操作性や利便性,文字・画像の表示方法の適切さを評価する実験を行い,その結果を考察する.

\subsection{実験内容}

この実験では,4章で実装したシステムを被験者に装着させ,読書をさせる.読書前にデバイスの操作方法の説明,LeapMotionによる入力ジェスチャの練習を約5分間行う.その後被験者はデバイスの機能を利用しながら各々のペースで紙の書籍の小説を約20ページ読む.その際に,Wikipediaでの語句の意味の検索機能とgoogleでの関連画像検索機能をそれぞれ最低でも1回以上使用させる.そして読書の終了後にデバイスの操作感や情報付与の方法についてアンケートにより調査する.項目は15項目あり,内容は以下のようになっている.読書中の手元の書籍の紙面は,アンケートの結果を考察するため,ビデオカメラによって記録した.

\subsection{実験結果}

実験のアンケート調査で得られた評価を表\ref{survey}に示す.
この結果によると,プロジェクションマッピングしている文字の表示方法に関して,文字の表示位置(質問1)に対しての意見は賛否が分かれる結果となった.文字の大きさ(質問3)については適切であると答える被験者が多かったが,被験者Bと被験者Dは適切でないと回答した.しかし文字の明るさに関しては平均値が4近くであり,概ね適切であると回答する被験者が多かった.
また画像の表示位置(質問5)についても賛否が分かれる結果となった.表示画像の大きさ(質問7)に関しては,全体として平均値が2近くの評価が得られ,あまり適切ではなかったと考えられる.しかし画像の明るさ(質問8)については,文字の明るさと同様概ね適切であるという意見が得られた.
また,文字や画像の表示タイミングについてはあまり良い結果は得られなかった.なお,プロジェクタの照射範囲(質問10)に関しては,平均値が3.8と概ね適切であるという結果が得られた.またデバイスの利便性(質問11)に関しては,3以下の評価が多く,これは質問13のデバイスの操作性に関しての評価が低かったこととも関連があると推察される.


\subsection{考察}

アンケートでの質問の回答についてそれぞれ考察する.
まず質問1については,トラッキングの精度による影響が大きかった.トラッキングが上手くいっている場合では,表示位置が適切であったという意見が得られた反面,トラッキングがずれていた場合では本の文章の文字と投影している文字とが重なり,文字が読みづらかったという意見が得られた.なお文字が重なってしまっているが,投影している文字が明るいため,読みづらさは感じなかったという意見もあった.しかし低い評価の理由には,トラッキングの影響以外にも,プロジェクタの照射範囲が本のページよりも狭く,投影する文字が見切れてしまっているという意見もあった.さらに語句を検索し,名詞を選択する際にデバイスの構造上左側に表示された名詞が選択しづらく,より右側に名詞を表示すべきだという意見も得られた.次に質問3の文字の大きさについては,適切であるという評価が多かった中で被験者Bと被験者Dが適切でないと答えたが.本実験では,投影する文字の大きさを書籍の文章よりも少し大きい程度の大きさで表示したため,投影文字が小さいと感じたのだと考えられる.質問5については,これもトラッキングの精度の影響があると考えらえる.トラッキングが上手くいっている被験者は文章と画像が被らずに見やすかったと回答していた.トラッキングがずれていた被験者に関しては,文字と画像が被っていたり,本の中央に画像が表示され見づらかったと回答していた.なおトラッキングが上手くいっていた場合でも,画像が見切れてしまい見づらかったという意見があった.質問7の回答に関しては,文字と画像が被ることを避けるために画像をやや小さめに表示したことで画像が見づらくなり,評価が低くなったと考えられる.
質問9に関しては,データの処理に時間がかかっているため,名詞を選択してから名詞の意味・関連画像を表示するまでの間にラグがあり,評価が低くなってしまったと考えられる.質問11については全体的に平均またはそれ以下の評価をつける被験者が多かった.その理由として主にあげられたのは,ジェスチャによる操作が難しく,慣れが必要であったため,デバイスの機能を十分に使用できなかったというものであった.デバイス自体が大きく,軽量化すべきだという意見もあった.しかし,その装着感は気になるほどではなかったという意見も得られた.逆に肯定的な意見も多く,分からない単語をすぐに調べられるのは便利であるという意見もあった.質問13に関しては,LeapMotionによる操作性の悪さを理由に低い評価を下している被験者が多かった.多く寄せられた意見として,ジェスチャを認識させるのが難しく,特にswipeジェスチャを上手く行えないというものがあった.LeapMotionが意図せずジェスチャを認識してしまったり,key tapジェスチャとswipeジェスチャを誤認識してしまうという場合もあった.デバイスにLeapMotionが斜めに設置されているため,操作が直感的でないという意見もあった.なおジェスチャをする際は,必然的に片手で本を持たないといけないため,それが負担になったり,読書を妨害していると感じる被験者もいた.
また感想としては,入力操作をスムーズに行うことができトラッキングの精度が上がれば嬉しい,調べる癖がついて良いなどの意見が得られた.一方でジェスチャによる入力は向いていない,片手で本を持つのは疲れるという意見があった.


\subsection{実験のまとめ}

本実験では,提案システムを実装したデバイスを被験者に装着させ紙の小説の読書を行わせた.実験の際には,実装されているWikipediaでの語句検索機能とGoogleでの画像検索機能をそれぞれ1回以上使用させた.その結果,表のような結果が得られ,カメラによるトラッキング精度の低さやLeapMotionによる操作の難しさなど,デバイスに実装している機能の改善すべき点が複数見つかった.しかし,ユーザビリティを向上させられれば,より利便性の高い読書を実現できるという意見を得られたため,トラッキング手法や入力インタフェースの改善を行い,デバイスのユーザビリティを向上させることが必要だと分かった.

\newpage
\section{まとめ}

% 本論文では,プロジェクションマッピングによる情報提示を行うウェアラブルデバイスにより,紙の書籍による読書体験を拡張するシステムを提案した.提案システムはウェブカメラから取得した紙面の画像を取得し,光学文字認識により紙面の文章データを得る.そして取得した文章を形態素解析により形態素に分け,名詞のみを抽出し,紙面に表示する.ユーザが紙面に表示された名詞を選択すると,その名詞の意味または関連画像が検索され,紙面上に表示される.このシステムを用いて読書することにより,読者は分からない名詞が出てきた際にも,読書を中断して辞書や電子書籍で調べることなく調べ物が可能で,より拡張された読書を体験できると考えられる.
% 提案システムが,実際に読書の利便性を高められているか,実装した機能について調査した.その結果,デバイスの機能に関しては便利だという意見が得られたが,文字・画像の表示位置,トラッキングの精度と入力インタフェースに改善すべき点が見つかった.
% 今後の課題として,トラッキング精度を改善し,より適切な位置へ情報を投影することが必要である.入力インタフェースとして用いたLeapMotionが使いづらいという意見が多く寄せられたため,読書行為を阻害しないインタフェースの検討が必要である.なおデバイスの情報処理に時間がかかり,文字・画像の投影タイミングが遅いという結果も得られたため,プログラム内容の簡略化やより高速な言語を使用してのシステム開発が必要だと考えられる.


\newpage
\addcontentsline{toc}{section}{\protect\numberline{}{謝辞}}
\markright{}
\section*{謝辞}

本研究を行うにあたり,日頃より御指導,御激励を賜り,数々の御教示を頂きまし
た塚本昌彦教授,寺田努教授に深甚なる謝恩の意を表します.また,本研究に関して多大なる御助言を頂きました磯山直也特命助教に厚く御礼申し上げます.そして日頃よりお世話頂きました塚本・寺田研究室馬場安希技術補佐員に深謝いたします.神戸大学工学部電気電子工学科に在学中,御教示,御激励頂いた神戸大学電気電子工学科の諸先生方に感謝すると共に,諸職員の方々に感謝いたします.日頃より数々の御助言を下さいました諸先輩方,快適な環境を作って頂いた研究室の皆様方,実験に協力頂いた被験者の方々に深く感謝いたします.特に研究活動に対する多くのアドバイスとサポートを頂いた工学研究科の大西鮎美氏,西垣佑介氏,近藤杏祐氏,東南颯氏,正月凌介氏に深く感謝いたします.

\newpage
\addcontentsline{toc}{section}{\protect\numberline{}{参考文献}}
\markright{}
\begin{thebibliography}{99}
	
	%thesis
	\bibitem{takano} 
	 高野健太郎, 大村賢悟, 柴田博仁: ``短編小説の読みにおける紙の書籍と電子書籍端末の比較,'' 情報処理学会第141回HCI研究会, Vol. 141, No. 4, pp.1--8 (Jan. 2011).
	
	\bibitem{shibata} 
	柴田博仁, 大村賢悟: ``答えを探す読みにおける紙の書籍と電子書籍端末の比較,'' 情報処理学会第141回HCI研究会, Vol. 141, No. 5, pp. 1--8 (Jan. 2011).
	
	 \bibitem{pablo}
	 P. Delgado, C. Vargas, R. Ackerman, and L. Salmer\'on: ``Don't Throw Away Your Printed Books: A Meta-analysis on the Effects of Reading Media on Reading Comprehension,'' {\it Journal of Educational Research Review}, Vol. 25, pp. 23--38 (Nov. 2018).
	 
	 \bibitem{inagawa1}
	稲川暢浩, 藤波香織: ``読書体験を豊かにするための本の拡張に関する研究,'' 情報処理学会第70回全国大会, pp. 131--132 (Mar. 2008).
	 
	 \bibitem{inagawa2}
	稲川暢浩, 藤波香織: ``読書中に低い妨害感で効果的に付加情報を伝達するための情報提示方法,'' 情報処理学会第135回HCI研究会, Vol. 2009, No. 12, pp. 1--8 (Dec. 2009).
	 
	 \bibitem{hashimoto}
	 橋本直己, 櫻井淳一: ``インタラクティブなプロジェクションマッピングの実現,'' 映像情報メディア学会誌, Vol. 67, No. 2, pp. 60--63 (Jan. 2013).
	
	\bibitem{isogaki}
	磯垣広野, K. S. Kyung, 澤野知佳, 福井 悠, 小林孝浩, 鈴木宣也, 関口敦仁: ``可動鏡を使ったマルチプロジェクションシステム -Ptolemy-,'' 情報処理学会シンポジウム論文集, Vol. 2007, No. 4, pp. 85--86 (Mar. 2008).
	
	\bibitem{doi}
	土井麻由佳, 宮下芳明: ``プロジェクションマッピングによる箏への演奏提示,'' 第23回インタラクティブシステムとソフトウェアに関するワークショップ論文集(WISS 2015), pp. 181--182 (Feb. 2015).
	
	
	\bibitem{gupta}
	S. Gupta and C. Jaynes: ``The Universal Media Book: Tracking and Augmenting Moving Surfaces with Projected Information,'' {\it Proc. of the ACM International Symposium on Mixed and Augmented Reality (ISMAR 2006)}, pp. 177--180 (Oct. 2006).
	
	\bibitem{suzuki}
	鈴木若菜, 竹田一貴, 外山託海, 黄瀬浩一: ``プロジェクタを用いた情報投影による印刷文書へのインタラクティブ性の付加,'' 電子情報通信学会技術研究報告, Vol. 111, No. 317, pp. 69--74 (Nov. 2011).
	
	\bibitem{ota}
	太田脩平, 寺田 努, 塚本昌彦: ``装着型プロジェクタと可動鏡による周辺状況を考慮した映像投影手法,'' エンタテインメントコンピューティングシンポジウム (EC 2012) 論文集, Vol. 2012, No. 1,  pp.1--7 (Mar. 2012).
	
	\bibitem{pranav}
	P. Mistry, P. Maes, and L. Chang: ``WUW - Wear Ur World - A Wearable Gestural Interface,'' {\it Proc. of the ACM International Conference on Human Factors in Computing Systems Extended Abstracts (CHI EA 2010)}, pp. 4111--4116 (Apr. 2009).
	
	\bibitem{chris}
	C. Harrison, H. Benko, and A. D. Wilson: ``OmniTouch: Wearable Multitouch Interaction Everywhere,'' {\it Proc. of the ACM symposium on User interface software and technology (UIST '11)}, pp. 441--450 (Oct. 2011).
	
	\bibitem{christian}
	C. Winkler, M. Broscheit, and E. Rukzio: ``NaviBeam: Indoor Assistance and Navigation for Shopping Malls through Projector Phones,'' {\it Proc. of the ACM Conference on Human Factors in Computing Systems (CHI 2011)}, (Jan. 2011).
	
	
	 \bibitem{karitsuka}
	  狩塚俊和, 佐藤宏介: ``プロジェクタ投影型ウェアラブル複合現実感システム, 情報処理学会研究報告コンピュータビジョンとイメージメディア (CVIM), Vol. 104, pp. 141--146 (Sep. 2003).
	  
	\bibitem{leap}LeapMotion, \url{https://www.leapmotion.com/ja/}.
	
	\bibitem{opencv}OpenCV, \url{https://opencv.org/}.
	
	\bibitem{google}Google Cloud Vision API, \url{https://cloud.google.com/vision/}.
	
	\bibitem{wikipedia}Wikipedia, \url{https://ja.wikipedia.org/wiki/}.
			
	\bibitem{mecab} 
	MeCab: MeCab: Yet Another Part-of-Speech and Morphological Analyzer,  \url{http://taku910.github.io/mecab/}.
	
	\bibitem{morimi}森見登美彦: 夜は短し歩けよ乙女, 角川書店 (2006).

	
\end{thebibliography}


\end{document}