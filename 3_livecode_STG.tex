\section{競技性・観戦性を拡張したプログラミングゲームの開発}
本稿では,プログラミング初学者の興味喚起を目的としたプログラミングゲームのプロトタイプについて紹介し,設計指針に基づくデザインの詳細や行った評価実験と見つかった課題,実験に関する考察について議論する.

\subsection{関連研究}

\subsection{設計指針}

提案システムの実装にあたり,初学者の興味関心を高めるために3つの設計指針を設けた.

\begin{enumerate}
	\item 観戦するにあたり,高度な専門知識を必要としない
	
	初学者が提案システムでの対戦を観戦するにあたり高度な専門知識を必要としてしまっては,利用の心理的ハードルを上げかねない.極力前提知識なしに理解し,楽しめるようにデザインする必要がある.
	
	\item 手間がかからない

	初学者にとってプログラミング学習をする際に環境構築や普段使用しない独自ソフトウェアのインストールはモチベーションを下げかねない.今回は提案システムをJavaScriptによって制御可能なWebアプリケーションとして実装し,初学者が実際にプログラミングを行わずとも見るだけで利用できるようにした.
	
	\item リアルタイムな駆け引き・アドリブを取り入れる

	従来のプログラミング教育コンテンツとは異なり,スポーツなどの観戦する競技で見られるようなリアルタイムな駆け引き・アドリブの要素を提案システムに組み込むことで観戦して楽しめるようなデザインをする.
\end{enumerate}




\subsection{提案システム}

\subsection{評価実験}

\subsection{課題と考察}




