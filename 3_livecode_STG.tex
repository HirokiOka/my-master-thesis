\section{競技性・観戦性を拡張したプログラミングゲームの開発}
本稿では,プログラミング初学者の興味喚起を目的としたプログラミングゲームのプロトタイプについて紹介し,設計指針に基づくデザインの詳細や行った評価実験と見つかった課題,実験に関する考察について議論する.

\subsection{関連研究}

\subsection{設計指針}

提案システムの実装にあたり,初学者の興味関心を高めるために3つの設計指針を設けた.

\begin{enumerate}
	\item 観戦するにあたり,高度な専門知識を必要としない
	
	初学者が提案システムでの対戦を観戦するにあたり高度な専門知識を必要としてしまっては,利用の心理的ハードルを上げかねない.極力前提知識なしに理解し,楽しめるようにデザインする必要がある.
	
	\item 手間がかからない

	初学者にとってプログラミング学習をする際に環境構築や普段使用しない独自ソフトウェアのインストールはモチベーションを下げかねない.今回は提案システムをJavaScriptによって制御可能なWebアプリケーションとして実装し,初学者が実際にプログラミングを行わずとも見るだけで利用できるようにした.
	
	\item リアルタイムな駆け引き・アドリブを取り入れる

	従来のプログラミング教育コンテンツとは異なり,スポーツなどの観戦する競技で見られるようなリアルタイムな駆け引き・アドリブの要素を提案システムに組み込むことで観戦して楽しめるようなデザインをする.
\end{enumerate}

\subsection{提案システム}

本研究で提案するシステムでは,プログラミング初学者の興味喚起をするためにいくつかの工夫を施した.今回実装したシステムは,プログラマ同士がリアルタイムにプログラミングを行うことでキャラクタを制御し,対戦するという対人形式のプログラミングゲームである.この対戦の様子を初学者に観戦させることで,初学者のプログラミングに対する興味を高める.プログラミングゲームとして実装した理由としては,キャラクタがプログラムによって動作するというゲームの視覚的な出力が,初学者にとってプログラムの出力を理解する助けになると考えたためである.ゲームジャンルとしては,シューティングゲームの体裁をとった.これはシューティングゲームが「敵の攻撃を避けて,敵を攻撃する」というプリミティブなゲームシステムであり,見ていて展開を理解しやすいと考えたためである.またゲームUIはLivecodeLabやHydraなどのビジュアルライブコーディング環境を踏襲し,ゲーム画面上にエディタを重畳することで,プログラムとその出力の双方を同時に見ることが可能なように設計した.またゲームは以下の3つのフェイズに分かれている.

\subsubsection{自機プログラミングフェイズ}
対戦が開始するとこのフェイズに移行する.ここでは予めエディタにプレイヤが操作するキャラクタのコンストラクタが記述されており,プレイヤはパラメータを書き換えることができる.具体的なパラメータにはappearance,life,clock,powerがある.appearanceはキャラクタの外見であり,文字列を指定できるため,絵文字などを使ってプレイヤの好きな見た目を選ぶことが可能である.lifeはキャラクタの体力であり,いわゆるHP(ヒットポイント)を表している.非負の整数を指定でき,この値が0以下になるとプレイヤはゲームに敗北する.clockはキャラクタが行動できる回数の多さを表しており,非負の整数を指定できる.プレイヤは後述する行動プログラミングフェイズにおいて自分のキャラクタを制御するプログラムを記述し対戦するが,その際に記述したプログラムは10秒間ループして実行される.このループのインターバルを決めるのがclockであり,値が大きいほどインターバルは短くなる.powerはキャラクタの攻撃力を表しており,これも非負の整数を指定できる.この値が大きいほど,自分が操作するキャラクタの攻撃が相手キャラクタに命中した際に削るlifeの値が大きくなる.双方のプレイヤが各パラメータを記述し終わると次の行動プログラミングフェイズに移行する.

\subsubsection{行動プログラミングフェイズ}
このフェイズに進むと,自機プログラミングフェイズで記述したコンストラクタを元に各プレイヤが操作するキャラクタのインスタンスが作成され,ゲーム画面が表示される.各プレイヤはプログラムをエディタに記述し,作成したキャラクタを操作する.プレイヤは条件分岐や繰り返しなど従来のJavaScriptの文法の他に独自に用意されたプロトタイプメソッドを使うことができる.用意したメソッドにはキャラクタを移動するメソッド(moveUp(), moveDown(),randomMove())とキャラクタが攻撃を行うメソッド(shot())などがある.またプログラム内で各キャラクタのパラメータを参照することもできる.両プレイヤがプログラムを記述し終わると次のゲームフェイズに移行する.なおこのフェイズでは相手プレイヤがどのようなプログラムを記述しているかは見ることができない.

\subsubsection{ゲームフェイズ}
このフェイズでは行動プログラミングフェイズで記述したプログラムが10秒間ループして実行され,ゲームが進行する.この段階で両プレイヤは相手プレイヤが記述したプログラムを閲覧することができる.このフェイズにおいて相手キャラクタを攻撃し,lifeの値を0いかにしたプレイヤの勝利となる.勝敗が決まらない場合はプログラム終了時の各パラメータを引き継いだまま行動プログラミングフェイズに戻り,再度プログラミングしゲームフェイズに移行するという過程を勝敗が決まるまで繰り返す.

\subsection{評価実験}

\subsection{課題と考察}




