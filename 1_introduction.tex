\section{はじめに}

高速にIT化の進む近年では,スマートフォンの普及を皮切りに,IoT,AI,クラウドコンピューティングといった単語がメディアで散見され,コンピュータが着実に人々の生活を満たしつつあると言える.このコンピュータが遍在する現代において,人とコンピュータの接点は圧倒的に多くなり,コンピュータを専門的に扱う人でなくともそれに触れて暮らすことが当たり前となり,コンピュータとの親和性を高めることが必要である.近年ではその高速な社会のIT化に伴い,プログラミングが重要性を増している.以前は「プログラミングはエンジニアや理系の学生がやるもの」というような,専門家だけの技能であるという認識が強かったが,プログラミングを英語のように一般教養とし,プログラミングに関わる仕事に就かない人もある程度理解していることが望ましいという風潮がある.今年からの小学校でのプログラミング必修化がその例である.

(ここに文科省の資料など)

プログラミングという行為は機能的()であるだけでなく,自身のアイディアをコンピュータを介して表現する手段でもあり,プログラミングスキルを向上させることは自己表現の可能性を広げることにつながる.(クリエイティブコーディング,メイカームーブメント).しかし昨今ではその機能にのみ焦点が当たりがちで,(ここでディスる).プログラミングの「楽しさ」が疎かにされがちである.プログラミングを学習する上で楽しさを感じることは重要である.松本らによるC言語プログラミングの授業では、学習の楽しさや、プログラミングへの興味が高いほど授業の学習率が高いことが示されている\ref{matsumoto}.
しかし最初から理解が容易で楽しさを感じやすい読み書きや運動とは異なり,プログラミングは楽しさを感じるまでにある程度の習熟を要する.よく書かれた小説などの文章やプロスポーツでのファインプレーには初心者でも心動かされるもので,そうした小説家やスポーツ選手への憧れが自ら書くことや体を動かすへの感心を高めるが,プログラミングの場合にはそのような憧れを持つ機会に乏しいのが現状である.

プログラミング初学者にプログラミングの楽しさを実感させるために設計された学習コンテンツは多数存在し,学校教育に導入されているものもいくつか存在する.しかしその多くは小・中学生など若い年代をターゲットに設計されているため,高校生や大学生,それ以上の年代にとっては楽しさを感じにくい場合がある.

本研究では従来のシステムがターゲットとした世代よりも上の世代をターゲットとし,プログラミング学習支援をするシステムを2つ開発した.

本研究ではプログラミング初学者のプログラミングに対する興味関心を高めるため.ゲーミフィケーションやゲームといったエンタテインメントの要素を組み込んだコンテンツを開発し,初学者を対象に運用を行うことでその効果を検証した.

本論文では以降,2章でゲーミフィケーションを用いたコードリーディング支援システムについて述べ,3章でプログラミング初学者の興味喚起を目的としたプログラミングゲームについて述べる.4章で本論文をまとめる.