\section{はじめに}

高速にIT化の進む近年では,スマートフォンの普及を皮切りに,IoT,AI,クラウドコンピューティングといった単語がメディアで散見され,コンピュータが着実に人々の生活を満たしつつあるといえる.コンピュータが遍在する現代において,人とコンピュータの接点は圧倒的に多くなり,コンピュータを専門的に扱う人でなくともそれに触れて暮らすことが当たり前となり,コンピュータとの親和性を高めることが必要である.

近年ではその高速な社会のIT化に伴い,プログラミングが重要性を増している.以前は「プログラミングはエンジニアや理系の学生がやるもの」というような,専門家だけの技能であるという認識が強かったが,プログラミングを英語のように一般教養とし,プログラミングに関わる仕事に就かない人もある程度理解していることが望ましいという風潮がある.国際的にもロシアでは2009年からプログラミング教育が導入され,英国では2014年から「Computing」というコンピュータサイエンス,情報技術,デジタルリテラシーの3分野からなる科目が導入されており\cite{survey},日本においても今年度から小学校でのプログラミング教育が必修化されている\cite{guide}.


プログラミングという行為は機能的であるだけでなく,自身のアイディアをコンピュータを介して表現する手段でもあり,プログラミングスキルを向上させることは自己表現の可能性を広げることにつながる.近年ではProcessingやopenFrameworks,Arduinoなどの登場により,プログラミングにより創造的な表現を生み出す「クリエイティブ・コーディング」やメイカームーブメントが流行し,コンピュータ・サイエンス等の知識がなくともプログラミングによってものづくりをし,発表できる土壌が整いつつある.

しかしプログラミング教育の現場やその学習教材では,プログラミングの機能にのみ焦点が当たりがちで,プログラミングの「楽しさ」が疎かにされていることが少なくない.プログラミングをする上で楽しさを感じることは,学習の観点からも重要である.松本らによるC言語プログラミングの授業では、学習の楽しさや、プログラミングへの興味が高いほど授業の学習率が高いことが示されている\cite{matsumoto}.

しかし最初から理解が容易で楽しさを感じやすい読み書きや運動とは異なり,プログラミングは楽しさを感じるまでにある程度の習熟を要する.よく書かれた小説などの文章やプロスポーツでのファインプレーには初心者でも心動かされるもので,そうした小説家やスポーツ選手への憧れが自ら書くことや体を動かすへの感心を高めるが,プログラミングの場合にはそのような憧れをもつ機会に乏しいのが現状である.

プログラミング初学者にプログラミングの楽しさを実感させるために設計された学習コンテンツは多数存在し,ScratchやViscuitなどのビジュアルプログラミング言語(VPL: Visual Programming Language)がその代表的な例である.これらは実際に学校教育で導入され,小・中学生のプログラミング教育に貢献している.しかしそれらは小・中学生など若い年代をターゲットに設計されているため,高校生や大学生,それ以上の年代にとっては楽しさを感じにくい場合がある.また実践的なソフトウェア開発においてはテキストプログラミング言語(TPL: Textual Programming Language)を用いる場合がほとんどであるため,VPLとTPLの間に乖離があることも問題である.

本研究では従来のシステムがターゲットとしている世代よりも上の世代のプログラミング初学者を対象に,エンタテインメントの要素を用いることでプログラミングに対する抵抗感を減らし,プログラミングを楽しんでもらうためのシステムの開発を目指した.具体的には初学者のコードリーディングを促進するアプリケーション,プログラミングゲームを用いてプログラミングに対する興味喚起を行うアプリケーションの2つを開発し,それぞれ初学者を対象に運用を行うことでその効果を検証した.


本論文では以降,2章で研究指針について述べた後,3章でゲーミフィケーションを用いたコードリーディング支援システムについて述べ,4章でプログラミング初学者の興味喚起を目的としたプログラミングゲームについて述べる.5章で本論文をまとめる.