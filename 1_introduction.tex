\section{はじめに}


昨今,自然言語での読み書きや健康のために行う運動のようにプログラミングを一般教養にしようとする動きがある.小学校におけるプログラミング教育の必修化がその最たる例で,将来プログラミングに関わる職業につかない人でもプログラミングに関する知識やその考え方をある程度身につける方が望ましいと考えられている.

しかし最初から理解が容易で楽しさを感じやすい読み書きや運動とは異なり,プログラミングは楽しさを感じるまでにある程度の習熟を要する.よく書かれた小説などの文章やプロスポーツでのファインプレーには初心者でも心動かされるもので,そうした小説家やスポーツ選手への憧れが自ら書くことや体を動かすへの感心を高めるが,プログラミングの場合にはそのような憧れを持つ機会に乏しいのが現状である.
またプログラミング初学者にプログラミングの楽しさを実感させるために設計された学習コンテンツは多数存在し,学校教育に導入されているものもいくつか存在する.しかしその多くは小・中学生など若い年代をターゲットに設計されているため,高校生や大学生などにとっては楽しさを感じにくい場合がある.

本研究ではプログラミング初学者のプログラミングに対する興味関心を高めるため.ゲーミフィケーションやゲームといったエンタテインメントの要素を組み込んだコンテンツを開発し,初学者を対象に運用を行うことでその効果を検証した.

本論文では以降,2章でゲーミフィケーションを用いたコードリーディング支援システムについて述べ,3章でプログラミング初学者の興味喚起を目的としたプログラミングゲームについて述べる.4章で本論文をまとめる.