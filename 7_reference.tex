\begin{thebibliography}{99}
	
	%はじめに
	\bibitem{survey}諸外国におけるプログラミング教育に関する調査研究, \url{https://www.mext.go.jp/a_menu/shotou/zyouhou/detail/__icsFiles/afieldfile/2018/08/10/programming_syogaikoku_houkokusyo.pdf}.
	\bibitem{guide}小学校プログラミング教育の手引(第三版), \url{https://www.mext.go.jp/content/20200218-mxt_jogai02-100003171_002.pdf}.
	\bibitem{matsumoto}松本 絵里子, 佐藤良樹, 佐藤瑞帆ほか: C言語の概念と実行過程を可視化するプログラミング学習用アプリケーションの開発,専修ネットワーク&インフォメーション, No.24, pp.15-26 (2016).
	\bibitem{okamoto}岡本雅子, 喜多 一: プログラミングの 「写経型学習」 における初学者のつまずきの類型化とその考察:, パイデイア: 滋賀大学教育学部附属教育実践総合センター紀要: memoirs of the Center for Educational Research and Training, Shiga University, 2014, 22, pp.49--53.
	
	%コードリーディング関連
	\bibitem{progate}Progate, \url{https://prog-8.com/}.
	\bibitem{gist}GitHub Gist, \url{https://gist.github.com/}.



	%GitHubを使った研究
	\bibitem{guzman}E.Guzman et al: Sentiment analysis of commit comments in GitHub: an empirical study, In Proceedings of the 11th Working Conference on Mining Software Repositories, 2014, pp. 352--355.
	\bibitem{nagano}永野真知, 早瀬康裕, 駒水孝裕, 北川博之: GitHubとStackOverflowにおけるユーザ行動の統一的な分析, 情報処理学会第79回全国大会, pp. 363–364 (Mar. 2017).
	\bibitem{shibatou}柴藤大介, 有薗拓也, 宮崎章太, 矢谷浩司: CodeGlass:GitHubのプルリクエストを活用したコード断片のインタラクティブな調査支援システム, 情報処理学会インタラクション, vol.2019, pp.159–16 (Mar. 2019).

	%ゲーミフィケーションを使った研究
	\bibitem{ichinose}一ノ瀬智浩, 畑秀明, 松本健一: ソースコード上の技術的負債除去を活性化させるゲーミフィケーション環境の開発, 情報処理学会関西支部支部大会講演論文集, vol.2016, (Sept. 2016).
	\bibitem{mitani}三谷将大, 寺田実: Webアプリケーションによるゲーミフィケーションを用いたプログラミング上達支援システム, 第27 回インタラクティブシステムとソフトウェアに関するワークショップ, (Sept.2018).

	% \bibitem{hikawa}樋川一幸, 松田滉平, 中村聡史: コミュニケーションチャネルに入り込む研究室実験BOTの提案と運用, 情報処理学会研究報告グループウェアとネットワークサービス,vol.3, pp. 1–7(Mar. 2019).

	%コードリーディングに関する研究
	\bibitem{omura}大村裕, 渡部卓雄: プログラム理解のためのコードリーディング支援ツールの提案と実装, 日本ソフトウェア科学会講演論文集, vol.31, pp.443–446, (Sep. 2014).
	\bibitem{ishio}石尾隆, 田中昌弘, 井上克郎: ソースコード上での情報タグ伝播によるコードリーディング支援, ウィンターワークショップ論文集, Vol.2008, pp.31–32, (Jan. 2008).

	\bibitem{uww2019}Ubiquitous Wearable Workshop, \url{http://cse.eedept.kobe-u.ac.jp/uww2019/}.

	%プログラミングゲーム関連

	%プログラミングを用いたエンタテインメント
	\bibitem{topcoder}Topcoder, \url{https://www.topcoder.com/}.
	\bibitem{codegolf}浜地慎一郎: Code Golf, \url{http://shinh.skr.jp/dat_dir/golf_prosym.pdf}.
	\bibitem{seccon}SECCON, \url{https://www.seccon.jp}.
	\bibitem{robocode}Robocode, \url{https://robocode.sourceforge.io/}.

	%プログラミング教育システム
	\bibitem{tpl}H.Tsukamoto et al: Programming education for primary school children using a textual programming language, In 2015 IEEE Frontiers in Education Conference (FIE), pp. 1--7.
	\bibitem{tsukamoto}H.Tsukamoto et al.: Textual vs. visual programming languages in programming education for primary schoolchildren, 2016 IEEE Frontiers in Education Conference (FIE), IEEE, 2016. p. 1--7.
	\bibitem{scratch}M.Resnick et al: Scratch: programming for all, Communications of the ACM, 52(11), pp. 60--67.
	\bibitem{viscuit}原田康徳: 子供向けビジュアル言語 Viscuit とそのインタフェース, 情報処理学会研究報告ヒューマンコンピュータインタラクション (HCI), 2005, pp.41--48.
	\bibitem{dolittle}S.Kanemune et al: Dolittle: an object-oriented language for K12 education, EuroLogo, 2005, pp. 144--153.

	%ライブコーディング環境
	\bibitem{livecodelab}LivecodeLab, \url{https://livecodelab.net/}.
	\bibitem{hydra}Hydra, \url{https://hydra.ojack.xyz/}.

	%プログラミングゲームを使った研究
	\bibitem{long}L.Ju: Just For Fun: using programming games in software programming training and education, Journal of Information Technology Education: Research, 2007, pp. 279--290.
	\bibitem{joshua}J.Shi et al: Pyrus: Designing A Collaborative Programming Game to Promote Problem Solving Behaviors, Proceedings of the 2019 CHI Conference on Human Factors in Computing Systems, No. 656, pp. 1–12 (May.2019).
	\bibitem{julian}J.Moreno: Digital competition game to improve programming skills, Journal of Educational Technology \& Society, 15(3), pp. 288--297.
	\bibitem{minakuchi}水口充: 成績評価のためのプログラミングゲームの設計と実践, 研究報告エンタテインメントコンピューティング(EC), vol. 2016, pp. 1–7(July.2016).

	\bibitem{mashitani}増谷海人, 赤澤紀子: 仮想現実を用いた初学者向けプログラミング学習システムの提案, 2018年度情報処理学会関西支部支部大会講演論文集,vol. 2018, (Sept.2018).

	\bibitem{esprima}Esprima, \url{https://esprima.org/}.
	\bibitem{complexity} T.J.MCCABE: A complexity measure, IEEE Transactions on software Engineering, 1976, pp. 308--320.
	\bibitem{escomplex}escomplex, \url{https://github.com/escomplex/escomplex}.

\end{thebibliography}