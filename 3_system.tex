\section{提案システム}

本章では,プロジェクタを用いたデバイスによって読者の必要な情報を紙面に表示することで,紙の本における読書を拡張する提案システムについて述べる.

\subsection{システム要件}

提案システムにおいて,読書の利便性を高める機能を実装した上で,読書を妨害しないことが重要である.またシステムの利用中に,ユーザが必要なタイミングで機能を利用できることが必要である.よってシステム要件は以下のようになる.

\begin{itemize}
	\item ユーザがシステムの利便性を感じる
	\item ユーザの任意のタイミングで情報を取得できる
	\item ユーザの読書を妨害しない
\end{itemize}

そこで本研究では,提案システムを実装したデバイスによって任意のタイミングで,読書の妨げにならない位置にプロジェクションマッピングによって情報付与を行えるようにする.またデバイスへの入力には,指による入力が自然で読書を妨害しないと考えハンドジェスチャ認識デバイスによる入力を導入した.


提案システムでは,紙の書籍にユーザの必要な情報をプロジェクションマッピングすることにより読書体験を拡張することを目的とする.提案システムの概要を図に示す.


本研究では分からない単語があった際に,読書を中断して辞書や電子端末で調べなければならない状況が多々あると想定し,それを解決し読書の利便性を向上させるため,文章中の名詞の中から選択したものの意味または関連画像を紙面の余白に表示する機能を実装する.提案システムはウェブカメラ,PC,ハンドジェスチャ認識デバイス,モバイルプロジェクタで構成される.提案システムでは,まずウェブカメラで紙面をトラッキングし,画像を取得する.ウェブカメラで取得した画像をPCで処理しユーザの必要な情報を作成する.そして作成した情報をプロジェクタにより紙面に表示する.デバイスに対する入力はLeapMotionによって指先のジェスチャを取得し行う.

\subsection{語句検索機能}


ユーザの必要な情報を動的に投影対象にプロジェクションマッピングするには,投影対象をトラッキングすることが必要である.本研究では,システムの使用開始時にウェブカメラから捉えた紙面の領域を長方形の枠により選択する.その後は指定した領域を追跡し続けることで,紙面をトラッキングする.トラッキングの様子を図に示す.


そしてハンドジェスチャ認識デバイスにより入力を認識した際に,読書中のページの文章をウェブカメラにより取得する.
取得した画像には読書中の本の紙面が含まれているため,その画像から光学文字認識により文章を取得する.
文字認識によって取得した文章を形態素解析することにより,名詞を抽出する.次にプロジェクタにより紙面に名詞を表示する.ユーザは表示された名詞の中から意味を検索したいものあるいは関連画像を得たいものを手を上下に動かすジェスチャ(key tapジェスチャ)によって選択する.そしてユーザに選択された名詞について検索する.このシステムでは指定した名詞の意味または関連画像を表示すると述べたが,名詞の意味についてはWikipedia\cite{wikipedia}で検索した内容を,関連画像についてはGoogleの画像検索によって得た内容を表示する.検索結果を表示している様子を図と図に示す.


Wikipediaで検索をする際は,検索結果のページを全て表示すると,情報量が多く読書を妨害してしまうため,検索する名詞のサマリーをWikipediaから取得し,紙面の余白領域へ投影する.
検索結果が「曖昧さ回避」となる名詞に関しては,候補となる名詞の最初に表示されたものを選択している.
また画像検索する際は,関連画像として最初に表示されたものを紙面の余白領域へ投影する.
これらの投影されている検索結果はユーザがkey tapジェスチャを行うことにより,任意のタイミングで削除できる.



