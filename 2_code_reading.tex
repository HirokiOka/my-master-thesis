\section{コードリーディング支援システムに関する研究}
本項では,プログラミング初学者向けに作成したエンタテインメントの要素を用いたコードリーディングシステムについて紹介し,その運用結果と課題について議論する.

\subsection{関連研究}

GitHubを活用した研究はいくつかある.

\subsection{設計指針}
システムを実装するにあたり以下の設計指針を設けた.
\begin{enumerate}
  \item 遊び(エンタテインメント)の要素を取り入れる
  コードリーディングはプログラミングにおいて重要なスキルの1つであるが,プログラムを読解する経験の少ないプログラミング初学者にとって,淡白なプログラムをただ読み込むことは難しい.よってエンタテインメントの要素を付与してプログラムを読ませることで,コードリーディングの心理的障壁を下げることを目指した.
  \item 日常的に使用するツールに組み込む
  アスリートが継続的なトレーニングを怠らないように,ある技能を高めるには日常的な鍛錬を要する.プログラミングにおけるコードリーディングでも同様であると考え,極力システム利用のハードルを下げるためにシステムを日常的に使用するデバイス・アプリケーションから使用できるように設計した.
  \item プログラムを読むことに専念させる
  プログラムの記述を促進するならば対象のソースコードを1つの言語に絞るべきという考え方もある.コードを書くことが目的であれば,複数の言語を同時進行させることは混乱を招き,デメリットが大きい.しかし,コードを読むことが目的であれば,上級者との会話のきっかけができる,将来的に適材適所で言語を使い分けることにつながる,飽きずにシステムを使用させ,コードの読解に興味を持たせることにつながるなどメリットが大きい,従って本システムでは,多様な言語に触れられる機能と1つの言語に読解を絞った機能を実装した.
\end{enumerate}

\subsection{提案システム}
\subsubsection{実装機能}
システムで使用できる2つの機能について説明する.
\begin{enumerate}
  \item クイズ機能
  この機能はユーザが使用した際にGitHubから取得したランダムなプログラムを表示する.ユーザは表示されたプログラムを読み,どのプログラミング言語で記述されたプログラムかを回答する.正解した場合は得点が得られ,取得した累計得点の多さによるランキングが表示される.
  \item 占い機能
  この機能をユーザが使用するとGitHubからランダムに得られたプログラムが表示され,ユーザはそれを読解し自分なりの解釈をすることで,その日の運勢を占う.表示されたソースコードの意味を解釈することが必要であるため,必然的にコードの読解が必要となる.この機能においては表示されるプログラムをJavaScriptで記述されたもののみにした.
\end{enumerate}
\subsection{プロトタイプシステムとケーススタディ}
初めに,提案システムをチームコミュニケーションツールであるSlackのbotとして実装した.システムを利用する際は,このbotを導入しているチャンネルでコマンドを入力することで各機能を使用できる.
このシステムを筆者が所属する研究室のSlackに導入し,研究室のメンバーを対象としたケーススタディを行った..研究室のメンバーは3名のプログラミング初学者であり,1週間自由に機能を使用してもらい,各コマンドの使用回数を調べた.またアンケートによって所感を調査した.

\subsection{課題と考察}

